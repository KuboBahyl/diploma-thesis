\newpage

\chapter{Conclusions}

In this Thesis, we have shown data reflecting the tuning fork oscillations in superfluid $ \He $ bath (at seven different velocities - $ 1.35\unit{K} $, $ 1.55\unit{K} $, $ 1.65\unit{K} $, $ 1.80\unit{K} $, $ 1.95\unit{K} $, $ 2.05\unit{K} $, $ 2.15\unit{K} $) as well as the second sound waves propagating through the resonator. The method of second sound attenuation showed that the only relevant parameter related with production of quantized vortices is the velocity amplitude of the fork tip and the (high enough) frequency of oscillation. We estimated the critical velocities for the fundamental and overtone modes to be $ v_{\ind{c}}^{\ind f} = 0.3 \pm 0.1 \unit{m/s}$, and $ v_{\ind{c}}^{\ind o} = 0.7 \pm 0.2 \unit{m/s} $, respectively, which should scale with the frequency as $ \propto \sqrt{\kappa\omega} $. We also confirmed that this scaling is consistent with the obtained critical velocities.

Using results from drag force measurements, a transition from linear to non-linear drag was clearly observed at different velocities for given temperatures. Scaling the force to drag coefficient and the velocity to oscillatory Reynolds number, both with respect to the normal component of $ \He $, revealed many hydrodynamic properties of the system. Starting with the fact that only the normal component contributes to the drag force in laminar flow, continuing with dynamical similarity of both fork modes and ending with critical $\text{Re}_{\delta_{\ind n}c}=7\pm 2 $, above which the non-linear drag was apparent.

We divided the temperature curves in two categories - those for which the non-linearity seemingly appeared at one certain velocity (strongly correlated with the production of quantized vortices), and those for which the onset arises when exceeding the critical $\text{Re}_{\delta_{\ind n}c}$. More detailed analysis showed that in the case of fundamental mode, we can safely claim that QT occurred first only for $ T = 1.35\unit{K} $. For the overtone mode, this scenario actually could not be confirmed for any temperature at all. But we have to still keep in mind that due to a relatively high sensitivity threshold and less-than-optimal quality of the second sound signal, it is, at this point, difficult to arrive at more convincing conclusions. 

Finally, we introduced the \textit{flow phase diagram} as a good illustrative tool showing each type of turbulence in particular \textit{zones} and each of the temperature curves as a set of parallel lines, intersecting the boundaries of \textit{zones} in different order.

\subsection*{Summary}

We summarize all the achieved results and conclusions (denoting "$ \checkmark $" as a positive contribution, "\textbf{?}" as an unresolved question and "$ \times $" as an aspect that should be improved in future) as follows:


\begin{itemize}

	
	\item[\checkmark] Fundamental and overtone resonance mode of submerged tuning fork in superfluid $ \He $ have been found ($ f_0^{\text{f}} = 6.38\unit{kHz} $, $ f_0^{\text{o}} = 40.00\unit{kHz} $).
	
	\item[\checkmark] Second sound has been generated and a wide frequency sweep measured. The $ 1^{\ind{st}} $ mode appeared to be the clearest and was used in further measurements.
	
	\item[\checkmark] Reliable formation of quantum turbulence above $ v_{\ind{c}}^{\ind{{f}}} = 0.3 \pm 0.1 \unit{m/s}$ (fundamental) and $ v_{\ind{c}}^{\ind{{o}}} = 0.7 \pm 0.2 \unit{m/s}$ (overtone) was found to be temperature independent.
	
	\item[\checkmark] The theory of critical velocity scaling $ v_{\ind{c}} \propto \sqrt{\kappa\omega}$ has been found in agreement with the experimental results.
	
	\item[\checkmark] We obtained data across 4 orders of magnitude in the dimensionless velocity (from $ 10^{-2} $ to 100), which is a remarkably wide range.
	
	\item[\checkmark] The oscillatory Reynolds number defined for normal component of superfluid $ \He $ as $\text{Re}_{\delta_{\ind n}} = \rho_{\ind n}\delta_{n} v/\eta$ proved to lead to the correct scaling of the drag forces and turned out to be a useful quantity in our analysis.
	
	\item[\checkmark] Onset of non-linear drag has been observed above $ \text{Re}_{\delta_{\ind n}c}=7\pm 2 $ for both fork modes.
	
	\item[\checkmark] QT definitely occurs before CT at $ 1.35\unit{K} $, while CT occurs first at $ 2.15\unit{K} $ and $ 2.05\unit{K} $. This means that a temperature must exist between these two values, where both types of turbulence are likely to be created at the same time. With the data available, we can estimate this temperature to be close to $ 1.52\unit{K} $ for the tuning fork used in this study.
	
	\item[\textbf{?}] Except for the outermost temperatures, we cannot say with certainty which type of turbulence appears first.

	\item[\textbf{?}] The true shape of the \textit{turbulence zones} is still not clear. Without better-resolved measurements, we have only a rough estimate based on the critical velocities and the oscillatory Reynolds number (which may themselves be affected by the sensitivity of the second sound resonator and the unsystematic behaviour of the tuning forks).
	
	\item[$\times$] The sensitivity and quality of second sound sensors should be improved to provide less deviation of critical velocities. This may be achieved, e.g., by using a shorter resonator without a strong connection to the helium bath as the current one had (a 1~mm diameter hole).
	
	\item[$ \times $] The unsystematic behaviour of the tuning fork ought to be eliminated. We believe that this is indeed possible, as other experiments with the same tuning fork (even measurements on the dilution fridge in the Prague Laboratory of Superfludity) have shown no signs of such behaviour. The first steps to be taken is a better isolation of the tuning forks and the entire second sound resonator from the helium bath, such as in a pressure cell or inside an enclosure filled by helium only through superleaks.
\end{itemize}

This topic (identification of the onset of the different types of turbulence) will definitely need much more experimental (and theoretical) work to clarify what are the conditions to make QT or CT and how these effects can interact.