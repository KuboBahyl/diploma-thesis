\documentclass[a4paper, 12pt]{report}

% Margins: left 40mm, right 25mm, up and down 25mm
% Dimensions: 210mm x 297mm
% (warning, Latex default from left: 1in~25mm)

\usepackage{fullpage}
\addtolength{\hoffset}{0cm} % extends height of page
\addtolength{\textwidth}{0.5cm} % extends lenght of written text

% Everything about page geometry:
%https://www.sharelatex.com/learn/Page_size_and_margins

%% PRINTED VERSION
% uncomment this below to render text only on pages on the right side - for printed version
%\let\openright=\clearpage

%% ENCODINGS
\usepackage[utf8]{inputenc}
%\usepackage[IL2]{fontenc}
\usepackage[english]{babel}


%% PACKAGES
\usepackage{amsfonts}
\usepackage{amsmath}
\usepackage{amssymb}
\usepackage[font={footnotesize}]{caption}
\usepackage{csquotes}
\usepackage{enumerate}
\usepackage{epigraph}
\usepackage{graphicx}
\usepackage{setspace} % setup various spaces
\usepackage{titlesec}
\usepackage[titles]{tocloft}
\usepackage[labelfont={sf,bf}]{caption}
\usepackage{url}
\usepackage{wrapfig} % image wrapper
\usepackage{hyperref} % hyperlinks


%% MACROS
\newcommand{\unit}[1]{\,\mathrm{#1}} % nice units
\newcommand{\ind}[1]{\mathrm{#1}}
\newcommand{\He}{{}^4\mathrm{He}} % special Helium
\renewcommand{\Re}{\mathrm{Re}} % Reynolds number
\newcommand{\eps}{\varepsilon} % standard epsilon
\newcommand{\<}{\langle} % closed left
\renewcommand{\>}{\rangle} % closed right
\renewcommand{\vec}[1]{\mathbf{#1}} % nice bold vector
\renewcommand{\dot}{\!\cdot\!} % scalar product
\newcommand{\todo}[1]{ {\bf !!TODO!!}\qquad{#1} } % TODO marker

% optimal spacings
\setlength{\parskip}{9pt}
\setlength{\parindent}{0pt}
\setlength{\baselineskip}{1.2\baselineskip} % extend distance under equations

% The following part convinces Latex to not mess with the chapters...
\makeatletter
\def\@makechapterhead#1{
  {\parindent \z@ \raggedright \normalfont
   \Huge\bfseries \thechapter. #1
   \par\nobreak
   \vskip 20\p@
}}
\def\@makeschapterhead#1{
  {\parindent \z@ \raggedright \normalfont
   \Huge\bfseries #1
   \par\nobreak
   \vskip 20\p@
}}\newpage
\makeatother

% Defining a non-numbered chapter (but included in Summary)
\def\chapwithtoc#1{
\chapter*{#1}
\addcontentsline{toc}{chapter}{#1}
}

% Changing font of subsection
\titleformat{\subsection}
{\large\sffamily\bfseries}
{\thesubsection}{1.4em}{}


%%%%%%%%%%%%%%%%%%%%%%%%%%%%%%%%%%%%%%%%%%%%%%%%%%%%%%%%%%%%%%%%%%%%%%%%%%%%%%
%% FRONT PAGES
%%%%%%%%%%%%%%%%%%%%%%%%%%%%%%%%%%%%%%%%%%%%%%%%%%%%%%%%%%%%%%%%%%%%%%%%%%%%%%


\begin{document}
% \pagestyle{empty}
%
% % Cover page
% \begin{center}
	\Large\sf
	Comenius University

	Faculty of Mathematics, Physics and Informatics
	\bigskip

	{
		\Huge \bfseries \sffamily DIPLOMA THESIS
	}

	\vfill

	\includegraphics[width=0.5\textwidth]{graphics/fmfi_logo.jpg}

	\vfill

	{
		\Huge\sf Jakub Bahyl
	}

	\vfill

	{
		\Huge \bfseries \sffamily
		\mbox{Drag Force Scaling}
		\vspace{1mm}
		\mbox{and Quantum Turbulence}
		\mbox{in Superfluid Helium}
	}

	\vfill

	\Large\sf
	Bratislava, 2018

\end{center}

\newpage

%
% % Front page
% \begin{center}
	\large\sf
	Comenius University

	Faculty of Mathematics, Physics and Informatics
	
	\bigskip

	{
		\Large \bfseries \sffamily 
		DIPLOMA THESIS
	}
	
	\vspace*{10mm}

	\includegraphics[width=0.5\textwidth]{graphics/fmfi_logo.jpg} 

	\vfill
	\vspace*{5mm}

	{
		\LARGE\sf 
		Jakub Bahyl
	}

	\vspace*{5mm}

	{
		\LARGE \bfseries \sffamily
		\mbox{Quantum Turbulence}
		\mbox{in Superfluid Helium Down to the}
		\mbox{Zero Temperature Limit}
	}

	\vfill

	Department of Low Temperature Physics, Charles University in Prague

	\vfill

	\begin{tabular}{rl}

	Study programme: & Condensed Matter Physics \\
	\noalign{\vspace{1mm}}
	Study programme number: & ??? \\
	\noalign{\vspace{1mm}}
	Supervisor of the thesis: & RNDr. David Schmoranzer, PhD. \\
	\noalign{\vspace{1mm}}
	Consultant of the thesis: & Mgr. Emil Varga \\

	\end{tabular}

	\vfill

	{
		\Large\sf Bratislava, 2018
	}

\end{center}

\newpage
%
% % Formal ENG
% \includegraphics[width=0.9\textwidth]{docs/zadanie_Bahyl_EN.pdf}
% \newpage
%
% % Formal SVK
% \includegraphics[width=0.9\textwidth]{docs/zadanie_Bahyl_SVK.pdf}
% \newpage
%
% % Acnkowledgements
% \chapter*{Acknowledgment}
\pagenumbering{gobble}
%%%%%%%%%%%%%%%%%%%%%%%%%%%%%%%%%%%%%%%%%%%%%%%%%%%%%%

First and foremost, I would like to express my sincere thanks to my supervisor, David Schmoranzer, for his continuous support during my master studies, his enthusiasm and immense knowledge, even under the worst circumstances.\\
Also, I would like to thank the rest of the superfluid research group: Emil Varga, Martin Jackson, Prof. L.Skrbek and Patrik Švančara, for their insightful comments and given opportunities.

Most of the Thesis was written during my daily visits of the 	\textit{VacuumLabs} company and various cafés in Bratislava, Prague and Žilina, such as \textit{Satori Stage, Foxford, Sweet beans, black., Vnitroblock, Cafedu} and \textit{Café Republika}, which I would like to acknowledge, for making the task a lot more lightweight and an enjoyable exercise.

My deep thanks also belong to my family and girlfriend for regularly rejecting my tendency to not finish the academic track, thus encouraging me to finish this Thesis.

Lastly, I would like to express my disappointment about faculty recommendation to use \textit{Mathworks} software company's products (Matlab), use of which extended all my data processing works into an abnormal time scale.

\vglue 0pt plus 1fill

\noindent
I declare that I carried out this master thesis independently, and only with the cited
sources, literature and other professional sources.


\vspace{10mm}

\hbox{
	\hbox to 0.5\hsize{%
		In \parbox[b]{2.5cm}{\dotfill}
		date \parbox[b]{2.5cm}{\dotfill}
	\hss}
	\hbox to 0.5\hsize{%
		Signature of the author:
		\hss}
	}

\newpage

%
% % Summary
% \vbox to \vsize{
\setlength\parskip{5mm}

\todo{}

{\bfseries \sffamily Názov práce}\\
Kvantová turbulencia v supratekutom héliu v teplotnej limite absolútnej nuly

{\bfseries \sffamily Autor}\\
Jakub Bahyl

{\bfseries \sffamily Školiteľ bakalárskej práce}\\
RNDr. David Schmoranzer, Ph.D.

{\bfseries \sffamily Abstrakt}\\
V tejto diplomovej práci prezentujeme meranie kvantovej turbulencie generovanej viacerými oscilujúcimi rezonátormi (\textit{kremenné oscilátory, vibrujúce drôty a torzné disky}) pri viacerých teplotách v škále  $1.0\unit{K} < T < 2.17\unit{K}$. Hlavná pointa experimentálnej časti je analýza vzniku kvantovej turbulencie a odporových síl normálnej a supratekutej zložky. Zozbierané dáta sú spolu s dátami z balistického režimu  $T < 0.6\unit{K}$ použité aj ako test tzv.
\textit{univerzálneho škálovania} popisujúceho odporové sily v režime vysokýchh Stokesových čísel oscilačného prúdenia.\\
Experimentálna časť je doložená numerickou implementáciou a validáciou programu, vyvinutého pre účel simulovania dynamiky kvantovaných vírov. V tejto časti skúmame použiteľnosť programu postaveného na princípe modelu \textit{Vírových filamentov}.


{\bfseries \sffamily Kľúčové slová}\\
supratekutosť $ \He $ $\bullet$ kvantový vír a turbulencia

\noindent\rule{16cm}{0.5pt}

{\bfseries \sffamily Title}\\
Quantum turbulence in superfluid helium down to the zero temperature limit

{\bfseries \sffamily Author}\\
Jakub Bahyl

{\bfseries \sffamily Supervisor}\\
RNDr. David Schmoranzer, Ph.D.

{\bfseries \sffamily Abstract}\\
In this Thesis, mechanical resonators such as \textit{tuning forks, wires are discs} are used to probe superfluid helium in the two-fluid regime at temperatures between temperatures $1.0\unit{K} < T < 2.17\unit{K}$. The principal aim of the experimental part is to study the drag forces arising from the superfluid and the nucleation and development of quantum turbulence. The obtained data, also with those from the ballistic regime $ T < 0.6\unit{K}$, are used to test a proposed \textit{Universal scaling relation} describing the drag forces in high-Stokes-number oscillatory flow.\\
The experimental work is complemented by developing and validating the tools necessary for numerical simulations of quantized vortex dynamics in superfluid helium. In this part, dynamics of a single quantized vortex ring are studied using the \textit{Vortex filament model}.


{\bfseries \sffamily Keywords}\\
superfluidity of $ \He $  $\bullet$ quantum vortex and turbulence

\vss}

\newpage
% \newpage

%%%%%%%%%%%%%%%%%%%%%%%%%%%%%%%%%%%%%%%%%%%%%%%%%%%%%%%%%%%%%%%%%%%%%%%%%%%%%%
%% THESIS
%%%%%%%%%%%%%%%%%%%%%%%%%%%%%%%%%%%%%%%%%%%%%%%%%%%%%%%%%%%%%%%%%%%%%%%%%%%%%%

\pagenumbering{arabic}
\pagestyle{plain}
\setcounter{page}{1}

% Content
\tableofcontents

% Chapters
\chapter*{Introduction / Motivation (3 pgs)}
\addcontentsline{toc}{chapter}{Introduction}

	liquid helium discovery, 1908, Heike Onnes, liquid state at 4.2K, superfluid state below 2.17K, full phase diagram:

	\begin{figure}[h]
		\centering
		\includegraphics[width=0.5\textwidth]{graphics/phase_diag}
		\caption{p-T diagram}
		\label{phase}
	\end{figure}

	labelling He-I, He-II, no solid state at 0K (weak van der Waals), only at 2.5MPa

	strange properties, thermal conductivity, negligible viscosity through capillaries

	Landau, Tisza: phenomenology, two-fluid model, proved bz rotating discs:

	\begin{figure}[h]
		\centering
		\includegraphics[width=0.5\textwidth]{graphics/densities}
		\caption{temperature dependence of densities}
		\label{densities}
	\end{figure}

	London: similarity of superfluid component with orbiting electorns, macroscopic wave func

	irrotational fluid, quantum vortices, tangle:

	\begin{figure}[h]
		\centering
		\includegraphics[width=0.5\textwidth]{graphics/QT-tangle}
		\caption{Quantum Turbulence}
		\label{QT}
	\end{figure}

	CT experiments: transition to turbulence, drag coeffs

	QT experiments: coflow, counterflow, second sound

	QT vs CT: complicated N-S equations, critical velocity or Reynolds number, QT has probably more critical velocities

	Simulations: filament model, boundaries

	Motivations: investigate critical velocities and vortex density, create numeric model

	Goals: measure hydrodynamic profiles for more temperatures with oscillating object, transition from CT to QT, investifate numerically vortex rings

\newpage
\chapter{Theoretical Background (15 pgs)}

The theoretical part of this Thesis is composed of two chapters:

\begin{itemize}
	\item[1.] Mesoscopic view - theoretically cover London's theory, creation and numerical modelling of quantum vortex, vortex dynamics.

	\item[3.] Macroscopic view - hydrodynamics of two-fluid model, oscillatory motion in such fluid, creation of QT, existence and usage of second sound

\end{itemize}

Many of this is covered in textbooks and papers.

He properties, total spin, Bose gas, critical temperature, heat capacity

\newpage

{\Huge \bfseries Mesoscopic view}
\addcontentsline{toc}{chapter}{Mesoscopic model}
\vspace{0.3cm}

\section{London's theory}
\begin{itemize}
	\item London's theory
	\item NLSE (Schr eq)
	\item macroscopic wave function
	\item no vorticity
	\item quantized circulation
\end{itemize}

\section{Quantum vortex}
\begin{itemize}
	\item definition
	\item induced velocity
	\item energy
	\item quantized circulation
	\item quantum turbulence
\end{itemize}

\section{Vortex filament model}
\begin{itemize}
	\item graph model
	\item state definition
	\item curve coordinates
	\item derivatives
	\item self-induced velocity
	\item LIA approximation
\end{itemize}

\section{Vortex dynamics}
\begin{itemize}
	\item magnus force
	\item mutual friction
	\item Schwarz's equation
	\item special case - quantum ring (properties)
	\item Kelvis waves (?)
\end{itemize}

\newpage

{\Huge \bfseries Macroscopic view}
\addcontentsline{toc}{chapter}{Macroscopic model}
\vspace{0.3cm}

\section{Hydrodynamics of two-fluid}
\begin{itemize}
	\item Landau's assumptions
	\item two densities, velocities (+pic)
	\item updated hydrodynamical equations - HVBK
	\item dynamical similarity
	\item Reynolds number
\end{itemize}

\section{Oscillatory motion in superfluid}
\begin{itemize}
	\item penetration depth
	\item Re for oscillations
	\item defining depth and Re separately for normal and superfluid components
\end{itemize}

\section{Quantum turbulence}
\begin{itemize}
	\item critical velocity according to landau
	\item critical velocity scaling in oscillatory case
	\item T dependence of critical velocities (Bc. results)
\end{itemize}

\section{Second sound}
\begin{itemize}
	\item what it is
	\item velocity of second sound
	\item attenuation
	\item vortex line density estimate
\end{itemize}

\newpage

%\chapter{Experimental Approach (10 pgs)}

\section{Apparatus}
\begin{itemize}
	\item cryostat
	\item cooling system
	\item insert
	\item resonator
\end{itemize}

\section{QT Generators}
\begin{itemize}
	\item quartz tuning fork
	\item other oscillators
\end{itemize}

\section{QT Detection}
\begin{itemize}
	\item Second sound generating
	\item SS detection
\end{itemize}

\section{Measurement methods and Processing}
\begin{itemize}
	\item fork modes - fund, overtone
	\item SS modes - working with ??-th mode
	\item frequency sweeps
	\item amp sweeps
	\item constant drives with SS on/off
\end{itemize}

\newpage

%\newpage

\chapter{Results}

Here we present the experimental data obtained from the measurements of quantum turbulence. The tuning fork has been immersed in superfluid $\He$ and forced to oscillate at two, geometrically different (in the sense of different velocity profile along the fork's prongs) modes - \textit{fundamental} $ [6380\unit{Hz}] $ and \textit{overtone} $ [40\,000\unit{Hz}] $.

The first part of this chapter is focused on the measurement of vortex line density $ L $ (total length of vortices in unit volume) using the second sound attenuation technique. We will try to find the conditions for production of quantized vortices and also, quantify their amount.

In the second part we will work only with the tuning fork via the applied voltage amplitudes and its current responses. Using the scaled drag coefficient and oscillatory Reynolds number, we will be able to estimate when the drag force acting on the tuning fork becomes non-linear with velocity. This event is a distinct sign of the flow pattern changing from simple laminar flow of the normal component and purely potential flow of the superfluid component, to something more complex involving some form of classical and/or quantum turbulence.

Overall we worked at seven selected temperatures in the two-fluid regime, when both superfluid and normal component amounts are considerable: $ 1.35\unit{K} $, $ 1.55\unit{K} $, $ 1.65\unit{K} $, $ 1.80\unit{K} $, $ 1.95\unit{K} $, $ 2.05\unit{K} $, $ 2.15\unit{K} $.  {\sffamily\textbf{Figure 3.1}} shows the time trace of the temperature inside the cell. Temperatures below $\approx 1.30\unit{K} $ were not stable, so the lowest fixed value was set to $ 1.35\unit{K} $.  


\begin{figure}[h!]
	\centering
	\includegraphics[width=1\textwidth]{graphs/diary}
	\caption{Record of the temperature inside the cryostat. Due to strong evaporation of superfluid helium and relatively long durations of the measurements at each temperature, we had to refill the cryostat three times. Unfortunately, this may have caused some unsystematic behaviour of the tuning fork that manifested after the first refill on Day 8, see the text for details.}
\end{figure}

Due to the prolonged experimental times, we had to refill helium three times, which has likely contributed to unsystematic behaviour of the tuning fork. The fork has yielded different results at the same temperature before and after transfers, especially the refill on Day 8. The data presented in this Thesis should thus be considered as if consisting of two separate categories of measurements: (i) temperatures $ 1.35\unit{K} $, $ 1.55\unit{K} $, $ 1.65\unit{K} $ investigated before Day 8; and (ii) remaining higher temperatures investigated after said transfer. Where we had the choice, we chose to present data measured before the transfer, as we believe that these better reflect the actual properties of the tuning fork and are more useful for comparison (e.g. they compare better with Ref. \cite{lancaster}). Similar effects in terms of different behaviour after the refilling have been observed with other oscillators such as vibrating wires \cite{history}.

We believe that the most likely scenario is that during the transfer and subsequent cooling, some small amount of air entered into the helium bath, solidified, and despite the protection offered by the resonator body, some bits of frozen air found their way towards the surface of the fork and were deposited there. Consequently, the altered tuning fork geometry could have affected the helium flow, especially the exact moment when the drag force becomes non-linear due to a flow instability. While this unfortunate event means that we are, in principle, working with two different oscillators rather than the exact same tuning fork at all of the investigated temperatures, we believe that the vortex line density measurements correlated with the drag forces acting on the tuning fork still provide very interesting results and are highly valuable not only as a basis for further studies and improvements, but also directly, as a study of oscillatory flows in superfluid helium (regardless of the exact geometry of the vibrating object).


\section{Measurement of Vortex Line Density}

It has already been shown that an object oscillating with sufficient velocity can produce quantized vortices in superfluid helium. For this purpose, we used the tuning fork mounted in the second sound resonator. Our measurement protocol is given in the following steps:

\begin{itemize}
	\item[1)] First, after the desired temperature in the cryostat has been reached, we run the frequency sweep on tuning fork and second sound independently. The tuning fork frequency sweeps are repeated at different drive levels. This gives us the necessary information about the resonance frequencies and widths.
	
	\item[2)] Next we set up the second sound sensors in constant drive mode at its fundamental resonance and allow up to 3 minutes for stabilization.
	
	\item[3)] When this time has passed, we also run the tuning fork in constant drive mode at its resonance (fundamental or overtone) with a given voltage amplitude $ U_0 $ for 3 minutes.
	
	\item[4)] The tuning fork is subsequently turned off and the second sound is again left to stabilize for 2 minutes.

	\item[5)] The values of $ A $ and $ A_0 $ were taken as averages of the periods when the tuning fork was on and off (cutting the transients), respectively (see {\sffamily\textbf{Figure 3.2}}).
\end{itemize}


\begin{figure}[h!]
\centering
\includegraphics[width=0.75\textwidth]{graphs/Attenuation.pdf}
\caption{An example of second sound attenuation due to the presence of quantized vortices, produced by an oscillating tuning fork at various velocities. "ON" and "OFF" labels describe the state of the tuning fork. The time on the x-axis is measured from the beginning  of each particular run. Values shown in this graph are taken at the temperature $ T=1.95\unit{K} $. The measurement at the velocity of $0.29\unit{m/s}$ corresponds to a vortex line density $L_0 = 3 \times 10^6 \unit{m}^{-2}$ and is taken as an estimate of the sensitivity threshold of our measurement technique.}
\end{figure}

The five aforementioned steps were repeated for several values of voltage applied across the tuning fork, for both fundamental and overtone mode at all six temperatures (all of the above except $ 1.65 \unit{K}$). We have always proceeded from low values of driving voltage to higher ones gradually, so that it is clear, at which point any measurable amount of quantum vortices appears.

By collecting datasets of $ A $, $ A_0 $ and $ \Delta f_0 $ we could estimate the vortex line density $ L $:

\begin{equation}
L = \frac{6\pi \Delta f_0}{B\kappa}\bigg( \frac{A_0}{A} - 1 \bigg)\,,
\label{eq}
\end{equation}

and the fork tip velocity $ v=I/a $, where $ I $ is current response and $ a $ the fork constant. The resulting plot ({\sffamily\textbf{Figure 3.3}}) is shown below.

We should point out that the results for $ L $ as derived in {\sffamily\textbf{Section 1.6}} are valid only for homogeneously and isotropically distributed vortices. The amount of quantized vortices is expected to be higher near the fork than further away from it. Since we utilized the $ 1^{\ind{st}} $ second sound resonant mode, we have, in fact, measured the $ 1^{\ind{st}} $ Fourier component of the vortex line density spatial distribution\cite{Emilfluids}. This is sufficient for the purposes of (roughly) estimating the quantities of quantized vortices produced, but the true values of $ L $ near the tuning fork may differ by some factor and could be obtained only by measurements using several additional second sound resonant modes.

\begin{figure}[h!]
	\centering
	\includegraphics[width=1\textwidth]{graphs/Merged_L_v(abs)}
	\caption{Vortex line density $ L $ against the (logarithmically scaled) peak velocity of the tuning fork $ v $. The \textit{blue dotted line} marks the threshold level $ L_0 \approx 3\cdot 10^6 \unit{m}^{-2} $ introduced in {\sffamily\textbf{Figure 3.2}}, above which the measured vortex line density can be regarded as reliable.}
\end{figure}


Nevertheless, from {\sffamily\textbf{Figure 3.3}} we observe that no significant amounts of vortex lines are produced before a certain critical velocity is exceeded. Furthermore, we find that the amount of quantized vortices produced is temperature-independent and scales only with the velocity of the tuning fork.

Plotting $ L $ on a logarithmic scale we observe (see {\sffamily\textbf{Figure 3.4}}) that the critical velocities, above which the quantum vortices are produced in much larger amounts, are also independent of temperature.
Bearing in mind the sensitivity threshold, we estimate these critical velocities for the fundamental and overtone modes to be $ v_{\ind{c}}^{\ind f} = 0.3 \pm 0.1 \unit{m/s}$, and $ v_{\ind{c}}^{\ind o} = 0.7 \pm 0.2 \unit{m/s} $, respectively. Moreover, as noticed in {\sffamily\textbf{Section 1.10}}, the critical velocity should scale with frequency as $ \propto \sqrt{\kappa\omega} $. From our results we get $ v_{\ind{c}}^{\ind f}/ v_{\ind{c}}^{\ind o} \cdot \sqrt{f_0^{\ind{o}}/f_0^{\ind{f}}} \doteq 1.06 $, which is consistent with the given scaling.

\newpage

\begin{figure}[h!]
	\centering
	\includegraphics[width=1\textwidth]{graphs/Merged_L_v(log)}
	\caption{Log-log graph of the vortex line density $ L $ against the peak velocity of the tuning fork (the same data as in {\sffamily\textbf{Figure 3.3}}). This graph better illustrates the position of threshold $ L_0 \approx 3\cdot 10^6 \unit{m}^{-2} $ (\textit{blue dashed lines}) and the temperature-independent critical velocities (\textit{grey dashed lines}). The tuning fork peak velocity is determined with an uncertainty of about $ 10\%$ that arises from the electrical calibration procedure\cite{opticalfork}. The vortex line density is affected by a systematic error that is mainly due to the assumption of a homogeneous isotropic tangle in deriving (\ref{eq}) that obviously does not correspond to the vortex tangle produced in the vicinity of the tuning fork.}
\end{figure}


Increasing the sensitivity of the second sound measurement (and thus lowering the threshold level) would allow determining the critical velocities with better accuracy and reduce the scatter in the observed data. Although the design of the second sound resonator and sensors is far from perfect, the current sensitivity is sufficient for a qualitative discussion of the relationship between the obtained vortex line densities and the drag forces acting on the tuning fork.


\newpage

\section{Drag Force Measurements}

All of drag force measurements presented in this Section were done solely with the tuning fork; second sound was not used during the measurement. We have operated the tuning fork in full frequency sweeps, with gradually increasing/decreasing driving voltages. Each of the datapoints in the following graphs is thus obtained from a full frequency sweep of tuning fork around its fundamental or overtone resonance frequency with a given applied voltage $ U_0 $. From such a sweep, the current response $ I $ was measured and for each pair $ \big[U_0,I\big] $ we found the corresponding values of peak applied force and peak velocity $ \big[F,v\big] $ using the calibration formulae from \cite{forks}: $ F = a U_0/2 $, $ v = I/a $.

\begin{figure}[h!]
	\centering
	\includegraphics[width=1\textwidth]{graphs/Merged_v_F}
	\caption{Force-Velocity characteristics for the fundamental and overtone modes of our tuning fork at seven different temperatures spanning the two-fluid regime. Apparently, a transition from linear to non-linear drag force occurs at high enough velocities. The \textit{blue dotted lines} sketch the approximate theoretical dependencies for laminar and classical turbulent flow.}
\end{figure}

In {\sffamily\textbf{Figure 3.5}} a transition from linear to non-linear drag is clearly observed. In classical fluid mechanics, the onset of force non-linearity is interpreted as a formation of a wake past the bluff body that encompasses vortical motion of the fluid and may eventually lead to the generation of turbulence. Our case is more complicated since we deal with two (possibly interacting) fluids and hence with two mutually-dependent types of turbulent motion.

It is also apparent that the linear drag force strongly depends on temperature, which reflects the fact that only the normal component contributes to the drag force in laminar flow. At the same time, the superfluid component exhibits purely potential flow, as an ideal fluid might, which results in a zero net contribution to the drag force as per d'Alembert's paradox \cite{landau}. To discuss the scaling properties in greater detail, it is, however, necessary to convert the force and velocity into relevant dimensionless quantities, such as the drag coefficient $ C_{D} = 2F/S\rho v^2 $ and the oscillatory Reynolds number $ \text{Re}_{\delta} = v\rho\delta/\eta $, , where $ S $ stands for the cross-section area of fork perpendicular to the direction of oscillation and $ F $ for the measured drag force.


\begin{figure}[h!]
	\centering
	\vspace{-0.3cm}
	\includegraphics[width=1\textwidth]{graphs/Merged_C_v}
	\caption{Drag coefficients plotted against the peak velocity of the oscillating tuning fork. The \textit{red dotted lines} fit the laminar part of the dependences for the temperature $ 1.35\unit{K} $, with different values of the fitting parameter for the two tuning fork modes. }
\end{figure}

Looking at the graph in {\sffamily\textbf{Figure 3.6}} we can recognize ``two types'' of curves. Those for which the non-linearity seemingly appears at one certain velocity, and those for which the onset arises when exceeding different velocities. This can be regarded as the first sign that either quantum turbulence (QT) or classical turbulence (CT) may occur first under different conditions. Moreover, we note that the $ 2.15 \unit{K}$ curve seems to approach the dependence of the drag coefficient as we know it from classical fluids, with the limiting constant value not far below $ C_D = 1 $, which is expected for tuning forks in a classical fluid~\cite{PraguePRB}.

Within the two-fluid model, we distinguish between the drag coefficient and oscillatory Reynolds number for the whole fluid $ C_{D}$, $ \text{Re}_{\delta} $ and only for the normal component alone $ C_{D{\ind n}} = 2F/S\rho_{\ind n}v^2 = C_D \rho/\rho_{\ind n} $, $ \text{Re}_{\delta_{\ind n}} = v\rho_{\ind n}\delta_{\ind n}/\eta_{\ind n} $. For each temperature, the values for the density of normal component $ \rho_{\ind n} $ and dynamical viscosity were taken from \cite{donnelly}.

The resulting graph ({\sffamily\textbf{Figure 3.7}}) showcases three interesting and important facts. First, within the laminar range, all curves collapse to a single dependence, confirming the validity of the scaling proposed in {\sffamily\textbf{Sections 3.3, 3.4}} and establishing the oscillatory Reynolds number as a suitable dimensionless parameter. Second, for both modes, the laminar part could be fit by the exact same straight line. This represents the fact that the drag force scaling in the high-frequency regime is, with all likelihood, independent on the velocity profile along the prong and holds even for frequencies differing by a factor of $40\,000/6380\approx 6.27 $. Finally, the critical Reynolds number for generating classical turbulence within both fork modes (the $ 2.15\unit{K} $ curve best shows) has been estimated as $ \text{Re}_{\delta_{\ind n}c}=7\pm 2 $.




\begin{figure}[h!]
	\centering
	\includegraphics[width=1\textwidth]{graphs/Merged_Cn_Ren}
	\caption{Drag coefficient plotted against the oscillatory Reynolds number, both evaluated for the normal component alone. Statistically, the onset of non-linear drag occurs for all temperature curves within the interval $ \text{Re}_{\delta_{\ind n}}^{\ind{crit}}=7\pm 2 $. The presented data are, however, affected by the observed unsystematic behaviour of the tuning forks with regard to the onset of non-linear drag, as discussed in the accompanying text. The \textit{red dotted line} shows the fit of the laminar regime, which is exactly the same for both modes.}
\end{figure}

At this point, we have to remind the reader that during the actual experiment, we have observed unsystematic behaviour of the tuning forks and therefore the data shown here is very likely affected by this issue. In drag force measurements, this manifested in such a manner that the onset of non-linear drag occurred at different values of $ \text{Re}_{\delta_{\ind n}} $ when the same temperature was studied at different points during the entire experiment. Therefore, the estimated critical value of $ \text{Re}_{\delta_{\ind n}c}=7\pm 2 $ ought to be taken with some reservations, as it corresponds only to a subset of the acquired data (the four higher temperatures), for which we have reasonable grounds to claim that no significant change of the tuning fork behaviour occurred in between the measurements. Nevertheless, the actual critical value (incorrect as it may be) has little bearing on the interpretation presented below, and we choose to use this particular value as it still describes the majority of the data with a good degree of accuracy.

\section{Correlation of Results}


In the following, we connect the results of vortex line density measurement and the non-linear behaviour of drag coefficient. We will be able to distinguish between quantum turbulence (QT), caused by the presence of quantized vortices and classical-like turbulence (CT) of the normal component. Within this Section, we work only with the fundamental mode.


\begin{figure}[h!]
	\centering
	\includegraphics[width=0.8\textwidth]{graphs/Merged_C+L_fund}
	\caption{Velocity dependence of drag coefficient $ C_D $ and vortex line density $ L $ for the fundamental mode of the tuning fork. The \textit{blue, red and black dotted lines} are the threshold level and laminar drag fits for $ 1.35\unit{K} $ and $ 1.80\unit{K} $ curves, respectively. The \textit{shaded zone} marks sub-critical velocities from the point of view of detection of quantized vortices by second sound, and corresponds to the laminar drag regime for the lowest three temperatures.}
\end{figure}


In {\sffamily\textbf{Figure 3.8}} we clearly see the correlation between the production of QT (lower graph) and the onset of non-linear drag force in $ 1.35\unit{K} $, $ 1.55\unit{K} $ and $ 1.65\unit{K} $ curves. For the other curves ($ 1.80 \unit{K} $ and above), the density of normal component is much higher and consequently the critical oscillatory Reynolds number was exceeded earlier than the critical velocity $v_{\ind{c}}^{\ind f}$.

However, from {\sffamily\textbf{Figure 3.8}} we are not sure, whether QT really occurred first or CT was just too weak to be noticed by our devices. The graphs in {\sffamily\textbf{Figure 3.9}} compare the normal drag coefficient $ C_{D\ind{n}} $ and vortex line density $ L $ against $ \text{Re}_{\delta_{\ind n}} $ and better illustrate the situation. We purposely zoomed the region near transition. 


\begin{figure}[h!]
	\centering
	\includegraphics[width=0.8\textwidth]{graphs/Merged_C+L_Ren_fund}
	\caption{Plot of the normal fluid drag coefficient $ C_{D\ind{n}} $ and vortex line density $ L $ against the oscillatory Reynolds number $ \text{Re}_{\delta_{\ind n}} $ for the fundamental mode of tuning fork.}
\end{figure}

As shown before, classical turbulence of the normal component arises roughly when exceeding the critical $ \text{Re}_{\delta_{\ind n}c} \approx 7$. On the other hand, QT has been detected above $ v_{\ind{c}}^{\ind{f}} \approx 0.3 \unit{m/s}$, so in {\sffamily\textbf{Figure 3.9}} it is clearly shown that a considerable QT was detected as first only for temperature $ 1.35 \unit{K} $. However, following the previous set of graphs in {\sffamily\textbf{Figure 3.8}}, we came to a similar conclusion for all three of the lowest temperatures ($ 1.35 \unit{K} $, $ 1.55 \unit{K} $, $ 1.65 \unit{K} $), not just the lowest one. This could be understood in such a manner that although CT could have occurred first for $ 1.55 \unit{K} $ and $ 1.65 \unit{K} $, QT followed shortly and soon became a dominant contribution to the drag force due to the high density of the superfluid component.

\newpage


\subsection*{Overtone Mode}

Here we perform exactly the same dependencies and similar conclusions we introduced for the fundamental mode in previous {\sffamily\textbf{Section 3.3}}.

\begin{figure}[h!]
	\centering
	\includegraphics[width=0.8\textwidth]{graphs/Merged_C+L_over}
	\caption{Velocity dependence of drag coefficient $ C_D $ and vortex line density $ L $ for the overtone mode of the tuning fork. The \textit{blue, red and black dotted lines} are the threshold level and laminar drag fits for the $ 1.35\unit{K} $ and $ 1.80\unit{K} $ curves, respectively. The \textit{shaded zone} marks sub-critical velocities from the point of view of detection of quantized vortices by second sound, and corresponds to the laminar drag regime for the lowest three temperatures.}
\end{figure}

The situation for the overtone mode seemingly does not differ from the fundamental - the only difference is that the correlation of QT formation and onset of non-linear drag is harder to see by the naked eye.

The plots of $ C_{D\ind{n}} $ and $ L $ against $ \text{Re}_{\delta_{\ind n}} $ (shown in {\sffamily\textbf{Figure 3.11}}) will help to illustrate the order in which the turbulence transitions likely occurred.


\newpage

\begin{figure}[h!]
	\centering
	\includegraphics[width=0.8\textwidth]{graphs/Merged_C+L_Ren_over}
	\caption{Plot of the normal drag coefficient $ C_{D\ind{n}} $ and vortex line density $ L $ against the oscillatory Reynolds number $ \text{Re}_{\delta_{\ind n}} $ for the overtone mode.}
\end{figure}

In contrast with the fundamental mode, here measurable QT did not occur earlier than CT at any of the shown temperatures. Although the graph in {\sffamily\textbf{Figure 3.10}} looked like the displayed non-linearities of the lowest three temperature curves are caused due to the formation of quantized vortices, this does not necessarily have to be true. Similarly to the fundamental mode, CT may have appeared earlier, but QT dominated rapidly and hence CT could not be seen in the drag coefficients reliably.

\newpage

\section{Discussion}

In this Section we summarize all the observations and found relations between the measured drag forces and the vortex line densities detected by second sound attenuation. We also put together a \textit{flow phase diagram} for better visualisation of the whole concept. To start with, we have already found the following:

\begin{itemize}
	\item[\textbf{I.}] \textbf{Vortex line density measurement} - A significant amount of quantized vortices are produced upon exceeding the temperature-independent critical velocities $ \approx 0.3 \unit{m/s} $ and $ \approx 0.7 \unit{m/s} $ for the fundamental and overtone modes of the tuning fork, respectively. In addition, the scaling factor for critical velocity generally agrees with the prediction based on quantized vortex dynamics $ \sim \sqrt{\kappa \omega} $.
	
	\item[\textbf{II.}] \textbf{Drag force measurement} - Until a certain oscillatory Reynolds number $ \text{Re}_{\delta_{\ind n}c}=7\pm 2 $ is exceeded, the drag force is linear with velocity. For higher temperatures, where there is considerably less superfluid component, there is a clear transition to a $ C_D \approx \text{const.} $ non-linear drag, similar as in classical fluids and likely related to turbulence of the normal component. 
	
	\item[\textbf{III.}] \textbf{Correlations} - The onset of non-linearity for $ 1.35\unit{K} $, $ 1.55\unit{K} $ and $ 1.65\unit{K} $ curves is definitely caused by the production of quantized vortices, although classical turbulence of normal component appeared earlier in the most of cases, but with negligible effect on drag. In other cases ($ 1.80 \unit{K}$ and above), CT dominates as the first.
\end{itemize}


\subsection*{Flow Phase Diagram}

To have a better idea, when each of the turbulences occur, we will attempt to construct a \textit{flow phase diagram}, plotting $ \text{Re}_{\delta_{\ind n}} $ against velocity and illustrating the areas of non-linear drag forces at each temperature. For the time being, let us consider a much simplified situation where CT and QT do not affect each other, i.e., no interactions between the normal and the superfluid component. Within this first approach, QT is created only  when the critical velocity $ v_{\ind{c}}^{\ind{{f}}} = 0.3 \pm 0.1 \unit{m/s}$ (for fundamental mode) or $ v_{\ind{c}}^{\ind{{o}}} = 0.7 \pm 0.2 \unit{m/s}$ (for overtone mode) is exceeded and CT forms when exceeding the critical oscillatory Reynolds number $ \text{Re}_{\delta_{\ind n}c}=7\pm 2 $. For a more direct comparison of the two resonant modes of the tuning fork, we additionally rescale the peak velocities by the factor of $\sqrt{\kappa \omega}$, obtaining a \textit{dimensionless velocity}. Whether CT and/or QT occur or not can be then shown using rectangular \textit{turbulent zones} in a flow phase diagram combining these two quantities.

Let us recall the definition of the oscillatory Reynolds number for normal component:

\begin{equation}
	\text{Re}_{\delta_{\ind n}}(T) = \frac{\rho_{\ind n}(T)\delta_{n}(T) v}{\eta(T)}
	= \sqrt{\frac{2\kappa\eta(T)}{\rho_{\ind n}(T)}}\cdot \frac{v}{\sqrt{\kappa \omega(T)}}\,,
	\label{phase_diag}
\end{equation}
where $ \eta(T) $ and $ \rho_{\ind n}(T) $ are temperature dependent viscosity and normal fluid density (experimental values can be found in \cite{donnelly}). According to (\ref{phase_diag}), we should get in a log-log flow phase diagram a set of parallel lines for the different temperatures, intersecting the boundaries of the \textit{turbulent zones} in different order.

Needless to say, this is a gross oversimplification of the real situation, as the presence of any significant amount of quantized vortices will lead to a non-negligible mutual friction force between the two components. At the same time, pressure and velocity fluctuations in any classical turbulence of the normal component may affect the precise moment at which quantized vortices pinned to the surface of the fork start to reconnect and multiply.

\begin{figure}[h!]
	\centering
	\includegraphics[width=1\textwidth]{graphs/FlowPhase_diagram}
	\caption{The simplified flow phase diagram illustrating the basic conditions for CT and QT to occur in the flow due to the investigated tuning fork. We note that only in the case of $ 1.35\unit{K} $ curve we are absolutely sure that QT was created first. The vertical and horizontal \textit{black dashed lines} marks the level of critical oscillatory Reynolds number $ \text{Re}_{\delta_{\ind n}c}\approx 7$ and dimensionless critical velocity $ v_{\ind c}^{\ind f}/\sqrt{2\pi \kappa f_0^{\ind f}} \approx v_{\ind c}^{\ind o}/\sqrt{2\pi \kappa f_0^{\ind o}} \approx 4.5$. }
\end{figure}

In reality, not only the transitions to CT and/or QT will affect (and possibly help trigger) one another, but one might also suspect that even the intensity of either type of turbulence will bear significant consequences for the other one, due to coupling of the two fluids via the mutual friction force. A more thorough analysis of these effects is clearly required and this topic will be subject of further experimental investigations in an improved realization of the experimental setup.

%\newpage

\chapter{Conclusions}

In this Thesis, we have shown data reflecting the tuning fork oscillations in superfluid $ \He $ bath (at seven different velocities - $ 1.35\unit{K} $, $ 1.55\unit{K} $, $ 1.65\unit{K} $, $ 1.80\unit{K} $, $ 1.95\unit{K} $, $ 2.05\unit{K} $, $ 2.15\unit{K} $) as well as the second sound waves propagating through the resonator. The method of second sound attenuation showed that the only relevant parameter related with production of quantized vortices is the velocity amplitude of the fork tip and the (high enough) frequency of oscillation. We estimated the critical velocities for the fundamental and overtone modes to be $ v_{\ind{c}}^{\ind f} = 0.3 \pm 0.1 \unit{m/s}$, and $ v_{\ind{c}}^{\ind o} = 0.7 \pm 0.2 \unit{m/s} $, respectively, which should scale with the frequency as $ \propto \sqrt{\kappa\omega} $. We also confirmed that this scaling is consistent with the obtained critical velocities.

Using results from drag force measurements, a transition from linear to non-linear drag was clearly observed at different velocities for given temperatures. Scaling the force to drag coefficient and the velocity to oscillatory Reynolds number, both with respect to the normal component of $ \He $, revealed many hydrodynamic properties of the system. Starting with the fact that only the normal component contributes to the drag force in laminar flow, continuing with dynamical similarity of both fork modes and ending with critical $\text{Re}_{\delta_{\ind n}c}=7\pm 2 $, above which the non-linear drag was apparent.

We divided the temperature curves in two categories - those for which the non-linearity seemingly appeared at one certain velocity (strongly correlated with the production of quantized vortices), and those for which the onset arises when exceeding the critical $\text{Re}_{\delta_{\ind n}c}$. More detailed analysis showed that in the case of fundamental mode, we can safely claim that QT occurred first only for $ T = 1.35\unit{K} $. For the overtone mode, this scenario actually could not be confirmed for any temperature at all. But we have to still keep in mind that due to a relatively high sensitivity threshold and less-than-optimal quality of the second sound signal, it is, at this point, difficult to arrive at more convincing conclusions. 

Finally, we introduced the \textit{flow phase diagram} as a good illustrative tool showing each type of turbulence in particular \textit{zones} and each of the temperature curves as a set of parallel lines, intersecting the boundaries of \textit{zones} in different order.

\subsection*{Summary}

We summarize all the achieved results and conclusions (denoting "$ \checkmark $" as a positive contribution, "\textbf{?}" as an unresolved question and "$ \times $" as an aspect that should be improved in future) as follows:


\begin{itemize}

	
	\item[\checkmark] Fundamental and overtone resonance mode of submerged tuning fork in superfluid $ \He $ have been found ($ f_0^{\text{f}} = 6.38\unit{kHz} $, $ f_0^{\text{o}} = 40.00\unit{kHz} $).
	
	\item[\checkmark] Second sound has been generated and a wide frequency sweep measured. The $ 1^{\ind{st}} $ mode appeared to be the clearest and was used in further measurements.
	
	\item[\checkmark] Reliable formation of quantum turbulence above $ v_{\ind{c}}^{\ind{{f}}} = 0.3 \pm 0.1 \unit{m/s}$ (fundamental) and $ v_{\ind{c}}^{\ind{{o}}} = 0.7 \pm 0.2 \unit{m/s}$ (overtone) was found to be temperature independent.
	
	\item[\checkmark] The theory of critical velocity scaling $ v_{\ind{c}} \propto \sqrt{\kappa\omega}$ has been found in agreement with the experimental results.
	
	\item[\checkmark] We obtained data across 4 orders of magnitude in the dimensionless velocity (from $ 10^{-2} $ to 100), which is a remarkably wide range.
	
	\item[\checkmark] The oscillatory Reynolds number defined for normal component of superfluid $ \He $ as $\text{Re}_{\delta_{\ind n}} = \rho_{\ind n}\delta_{n} v/\eta$ proved to lead to the correct scaling of the drag forces and turned out to be a useful quantity in our analysis.
	
	\item[\checkmark] Onset of non-linear drag has been observed above $ \text{Re}_{\delta_{\ind n}c}=7\pm 2 $ for both fork modes.
	
	\item[\checkmark] QT definitely occurs before CT at $ 1.35\unit{K} $, while CT occurs first at $ 2.15\unit{K} $ and $ 2.05\unit{K} $. This means that a temperature must exist between these two values, where both types of turbulence are likely to be created at the same time. With the data available, we can estimate this temperature to be close to $ 1.52\unit{K} $ for the tuning fork used in this study.
	
	\item[\textbf{?}] Except for the outermost temperatures, we cannot say with certainty which type of turbulence appears first.

	\item[\textbf{?}] The true shape of the \textit{turbulence zones} is still not clear. Without better-resolved measurements, we have only a rough estimate based on the critical velocities and the oscillatory Reynolds number (which may themselves be affected by the sensitivity of the second sound resonator and the unsystematic behaviour of the tuning forks).
	
	\item[$\times$] The sensitivity and quality of second sound sensors should be improved to provide less deviation of critical velocities. This may be achieved, e.g., by using a shorter resonator without a strong connection to the helium bath as the current one had (a 1~mm diameter hole).
	
	\item[$ \times $] The unsystematic behaviour of the tuning fork ought to be eliminated. We believe that this is indeed possible, as other experiments with the same tuning fork (even measurements on the dilution fridge in the Prague Laboratory of Superfludity) have shown no signs of such behaviour. The first steps to be taken is a better isolation of the tuning forks and the entire second sound resonator from the helium bath, such as in a pressure cell or inside an enclosure filled by helium only through superleaks.
\end{itemize}

This topic (identification of the onset of the different types of turbulence) will definitely need much more experimental (and theoretical) work to clarify what are the conditions to make QT or CT and how these effects can interact.
%% Norma citácii: ISO 690

\def\bibname{Bibliography}
\begin{thebibliography}{99}
	\addcontentsline{toc}{chapter}{\bibname}
	
\bibitem{kapitsa}
{\sc Kapitsa, P.}
\emph{The Super-Fluidity of Liquid Helium II}.
Nature \textbf{141}, 74 (1938)

\bibitem{tong}
{\sc Tong, D.}
\emph{Statistical physics}.
University of Cambridge Part II

	
	
	
\end{thebibliography}

\chapter{Experimental Approach (10 pgs)}

\section{Apparatus}
\begin{itemize}
	\item cryostat
	\item cooling system
	\item insert
	\item resonator
\end{itemize}

\section{QT Generators}
\begin{itemize}
	\item quartz tuning fork
	\item other oscillators
\end{itemize}

\section{QT Detection}
\begin{itemize}
	\item Second sound generating
	\item SS detection
\end{itemize}

\section{Measurement methods and Processing}
\begin{itemize}
	\item fork modes - fund, overtone
	\item SS modes - working with ??-th mode
	\item frequency sweeps
	\item amp sweeps
	\item constant drives with SS on/off
\end{itemize}

\newpage

\chapter{Simulations (10 pgs)}

\section{Finite differences}
\begin{itemize}
	\item FD order, radius
	\item Vandermonde vs analytical method
	\item comp complexities of LIA and Biot-Savart
	\item coords and velocities updating
\end{itemize}

\section{Integration }
\begin{itemize}
	\item Euler vs. RK4 step
	\item time stepping
	\item stability
\end{itemize}

\section{Resegmentation}
\begin{itemize}
	\item adding and removing segments
	\item local spline
\end{itemize}

\section{Vortex ring}
\begin{itemize}
	\item initialisation
	\item movement, decreasing radius
	\item comparison with theory
	\item Kelvin waves (?)
\end{itemize}

\newpage

\chapter{Results (15 pgs)}

\section{Vortex line density}
\begin{itemize}
	\item steps made for achieving results
	\item graphs for fundamental and overtone
\end{itemize}

\section{Drag force graphs}
\begin{itemize}
	\item velocity vs force
	\item C vs velocity
	\item Cn vs Reynolds
\end{itemize}

\section{Universal Scaling}
\begin{itemize}
	\item Donnely number
	\item universal scaling
\end{itemize}

\section{Correlations}
\begin{itemize}
	\item compare vortex generating with drag force graphs
	\item fund and overtone
\end{itemize}

\section{Simulations}
\begin{itemize}
	\item compare rings with various radii
	\item theoretical vs simulation velocity / range
	\item stability tests
	\item Kelvin waves(?)
\end{itemize}

\newpage

\chapter{Conclusions (2 pgs)}

\begin{itemize}
	\item summarize mainly what have we done
	\item repeat motivations and goals
	\item list of achievements
	\item list of failures
	\item list of improvements
	\item last words
\end{itemize}

\newpage

% Norma citácii: ISO 690

\def\bibname{Bibliography}
\begin{thebibliography}{99}
	\addcontentsline{toc}{chapter}{\bibname}
	
\bibitem{kapitsa}
{\sc Kapitsa, P.}
\emph{The Super-Fluidity of Liquid Helium II}.
Nature \textbf{141}, 74 (1938)

\bibitem{tong}
{\sc Tong, D.}
\emph{Statistical physics}.
University of Cambridge Part II

	
	
	
\end{thebibliography}


\end{document}
