In this work we present measurements of quantum turbulence generated by a $ 6.5\unit{kHz} $ oscillating quartz tuning fork submerged in superfluid $ \He $ at various temperatures below $ T_{\lambda} = 2.17\unit{K} $. The observed
non-linear drag acting on the tuning fork is qualitatively described by a presence of classical turbulence and quantized vortices. Drag forces and the amount of produced quantized vortices (the vortex line density $ L $) are determined indirectly by the attenuation of second sound and by the measurement of mechanical and electrical properties of the tuning fork.
We present results which characterize quantitatively the formation of classical and quantum turbulence. For the first time, we show that both turbulences can arise separately. This is presented, in first approximation, via a \textit{flow phase diagram}, where for a given temperature, one can predict when each type of turbulence forms.