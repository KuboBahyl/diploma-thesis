V tejto práci prezentujeme meranie kvantovej turbulencie generovanej oscilujúcim $ 6.5\unit{kHz} $ kremenným oscilátorom, ponoreným v supratekutom $ \He $ pri viacerých teplotách pod kritickou $ T_{\lambda} = 2.17\unit{K} $. Pozorovaná nelineárna odporová sila pôsobiaca na oscilátor je kvalitatívne spôsobená prítomnosťou klasickej turbulencie a kvantovaných vírov. Odporové sily a množstvo vytvorených kvantovaných vírov (hustota vírov čiar $ L $) sú nepriamo určené z útlmu druhého zvuku a mechanicko-elektrických vlastností oscilátora.
Výsledky, ktoré prezentujeme, kvantitatívne charakterizujú tvorbu klasickej aj kvantovej turbulencie. Ako prví ukazujeme, že obidve turbulencie vedia vzniknúť samostatne a nezávisle. Tento fakt je prezentovaný, v prvom priblížení, pomocou \textit{prúdového fázového diagramu}, v ktorom pre danú teplotu vieme predpokladať, kedy a ktorá turbulencia vzniká.