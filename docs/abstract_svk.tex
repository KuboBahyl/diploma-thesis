V tejto diplomovej práci prezentujeme meranie kvantovej turbulencie generovanej viacerými oscilujúcimi rezonátormi (\textit{kremenné oscilátory, vibrujúce drôty a torzné disky}) pri viacerých teplotách v škále  $1.0\unit{K} < T < 2.17\unit{K}$. Hlavná pointa experimentálnej časti je analýza vzniku kvantovej turbulencie a odporových síl normálnej a supratekutej zložky. Zozbierané dáta sú spolu s dátami z balistického režimu  $T < 0.6\unit{K}$ použité aj ako test tzv.
\textit{univerzálneho škálovania} popisujúceho odporové sily v režime vysokýchh Stokesových čísel oscilačného prúdenia.\\
Experimentálna časť je doložená numerickou implementáciou a validáciou programu, vyvinutého pre účel simulovania dynamiky kvantovaných vírov. V tejto časti skúmame použiteľnosť programu postaveného na princípe modelu \textit{Vírových filamentov}.
