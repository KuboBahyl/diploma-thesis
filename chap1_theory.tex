\chapter*{Introduction}
\addcontentsline{toc}{chapter}{Introduction}

	To this day, turbulent motion of fluids remains the last unresolved problem of classical physics. They present many practical challenges across different areas of industry (f.e. weather prediction). Quantum turbulence, in contrast, may occur only in superfluids and was first observed in superfluid state of $\He$. Compared to classical turbulence, it can be regarded as a simpler system from the theoretical and numerical point of view. Also, it shares many of the general properties of turbulence in classical viscous fluids.

	In very low temperatures, the liquid state of $\He$ exists in two phases:
	\begin{itemize}
		\item \underline{Helium-I}: a high temperature phase ($2.17\unit{K}<T<4.2\unit{K}$)
		\item \underline{Helium-II}: a low temperature phase ($T<2.17\unit{K}$)
	\end{itemize}

	These two phases are connected with the \textit{lambda transition}, which occurs at the critical temperature $T_{\lambda} = 2.17 \unit{K}$ at saturated vapour pressure (\textbf{Figure \ref{phase_diag}}). Helium-I is a classical fluid described by ordinary Navier-Stokes (N-S) equations, whereas Helium-II is a superfluid and behaves partly like a Bose-Einstein condensate.

	\begin{figure}[h]
		\centering
		\includegraphics[width=0.6\textwidth]{graphics/theory/phase_diag}
		\caption{Pressure-Temperature diagram of $\He$. Fixing pressure on an atmospheric value, a gas-liquid transition is present at $4.2\unit{K}$ (He-I) and a superfluid transition at $2.17\unit{K}$.}
		\label{phase_diag}
	\end{figure}

	A simple, phenomenological model of the Helium-II motion was proposed by Tisza \cite{tisza} and Landau \cite{landau} - the \textit{two-fluid model}. According to two-fluid model, it behaves as if composed of two inter-penetrating liquids - the normal and superfluid components - with corresponding velocity fields and temperature-dependent densities:

	\begin{itemize}
		\item \underline{normal component}: density $\rho_n (T)$, velocity field $\vec{v}_n (\vec{r}, t)$, motion described by an ordinary viscous Navier-Stokes equation, carrying entropy and represented as a collection of thermal excitations such as \textit{phonons} and \textit{rotons}
		\item \underline{superfluid component}: density $\rho_s (T)$, velocity field $\vec{v}_s(\vec{r}, t)$, motion described by a modified Euler equation (without viscosity) with quantum terms, no entropy and represented by a macroscopic wave function
	\end{itemize}

	The total density of Helium-II sums up to $\rho = \rho_n(T) + \rho_s(T) \approx \text{const}$ and the relative proportion of normal/superfluid component is determined mainly by the temperature (\textbf{Figure \ref{densities}}). Near $T \rightarrow 0$ Helium-II becomes entirely superfluid $\rho_s/\rho \rightarrow 1$. The temperature dependence of this ratio is highly nonlinear. For example, the ratio $\rho_n/\rho$ drops from $100\%$ at $2.17\unit{K}$ to $50\%$ at $1.95\unit{K}$, to $<5\%$ at $1.3\unit{K}$, and is effectively negligible under $1\unit{K}$.

	\begin{figure}[h]
		\centering
		\includegraphics[width=0.6\textwidth]{graphics/theory/densities}
		\caption{Temperature dependence of fractional densities of the normal (red) and superfluid (blue) components. Source: \cite{svoc2016}}
		\label{densities}
	\end{figure}

	It arises from the quantum nature of superfluid, that the superfluid component should not perform any rotation. However, when this component moves faster than a critical velocity, the circulation is \textit{quantized} and so-called \textit{quantized vortices} are created, which makes the hydrodynamics of Helium-II particularly interesting. The vortex nucleation process is still a subject of many current investigations. Superfluid vortex lines were observed spatially organized, but also completely disorganized as simulated in \textbf{Figure \ref{sim_cube}}. Quantum turbulence therefore takes the form of a tangle of quantized vortices in the superfluid component which typically coexist with classical turbulent flow of the normal component.

	In the presence of quantized vortices, the independent normal and superfluid velocity fields become coupled by the \textit{mutual friction} force which arises due to quasiparticles scattering off the cores of vortices.

	\begin{figure}[h]
		\centering
		\includegraphics[width=0.4\textwidth]{graphics/theory/QT-tangle}
		\caption{Cube of numerically simulated tangle of randomly distributed quantised vortices. Source: \cite{svoc2016}}
		\label{sim_cube}
	\end{figure}

	Also we note that for a typical experiment, below $\sim 0.7\unit{K}$, a transition to ballistic regime occurs in the normal component, as the mean free path of phonons exceeds the characteristic dimensions of the experimental setup. This situation is similar to the one of dilute classical gases.

	Quantum turbulence can be experimentally achieved in many traditional ways - driving a mass flow, spinning discs, oscillating grids and forks, ultrasound and jets.\\
	To characterize the turbulence one may use a superfluid Reynolds number for a steady flows, or a newly introduced \cite{universal_scaling} Donnelly number for high-frequency oscillatory flows.
	We find that for quantum turbulence originated in high-frequency regime above temperature $T > 1\unit{K}$, the measured drag forces are described in terms of a single dimensionless parameter and exhibit an universal scaling behaviour. We identify and compare the critical conditions related to the production of both quantized vorticity and instabilities occuring in normal component.

	Besides experimental approaches on quantum fluids, one of the useful tools for understanding the geometry and flow of quantum turbulence, is the \textit{vortex filament model}, pioneered by Schwarz \cite{schwarz}. With the rapid development of available computational power, large simulations have become the methods of choice for calculating the motion of fluids. In superfluids like Helium-II, due to the quantization of circulation, vorticity can only exist within vortex filaments with a certain core size, which makes the model highly applicable.

	We propose an efficient numerical method to compute the time evolution of vortex filaments in superfluid Helium-II. We studied the performance and stability and well replicated the physical processes such as the annihilation of quantized vortex rings \cite{vortex_ring} while travelling across superfluid.\\
	We also present the \texttt{PyVort} code, a new platform in Python 3 to simulate quantized rings phenomena. More on the implementation part can be found in \textbf{Simulation} and \textbf{Appendix}.

%%%%%%%%%%%%%%%%%%%%%%%%%%%%%%%%%%%%%%%%%%%%%%%%%%%%%%%%%%%%%%%%%%%%%%%%%%%%%%

	\section*{Motivations and Goals}

	Here we briefly collect all motivations and goals that led us to our investigations.

	\subsection*{Experimental approach}

	\begin{itemize}
		\item ivestigate transition from laminar and potential flow of normal and superlfuid components, respectively, to classical or quantum turbulence at various temperatures above $>1\unit{K}$ in high-frequency regime.
		\item construct an experiment using flow generators such as vibrating wire, tuning fork and oscillating disc and observe the drag phenomena
		\item apply universal scaling theory and prove the concept on collected experimental data
	\end{itemize}

	\subsection*{Simulation}
	\begin{itemize}
		\item build modular and reusable codebase in Python 3 that simulates the dynamics of quantized vortices using the \textit{Vortex Filament} model
		\item implement stable time-step methods and a reliable re-segmentation process that allows keeping a good resolution of quantized vortices in different situations
		\item simulate a real-time quantum vortex ring motion and compare its properties and evolution in time with the theoretical approaches, thus validating the new codebase
	\end{itemize}

\newpage

%%%%%%%%%%%%%%%%%%%%%%%%%%%%%%%%%%%%%%%%%%%%%%%%%%%%%%%%%%%%%%%%%%%%%%%%%%%%%%
%%%%%%%%%%%%%%%%%%%%%%%%%%%%%%%%%%%%%%%%%%%%%%%%%%%%%%%%%%%%%%%%%%%%%%%%%%%%%%
%%%%%%%%%%%%%%%%%%%%%%%%%%%%%%%%%%%%%%%%%%%%%%%%%%%%%%%%%%%%%%%%%%%%%%%%%%%%%%

\chapter{Theoretical Background}

The theoretical part of this Thesis is composed of two chapters:

\begin{itemize}
	\item[1.] \underline{Micro/Meso-scopic view} - provides a theoretical cover of Gross-Pitaevskii model, creation and numerical modelling of the quantized vortex and its dynamics.

	\item[3.] \underline{Macroscopic view} - provides a hydrodynamics description of two-fluid model, oscillatory motion in He-II, creation of quantum turbulence, dynamical and universal scaling principles

\end{itemize}

The aim of the theoretical part is to introduce the basic properties of quantized vortex lines in Helium-II and summarize the state of art of current knowledge on superfluid turbulence. We also discuss the theoretical methods used to study quantized vorticity, quantum turbulence and the results obtained using such methods.

%%%%%%%%%%%%%%%%%%%%%%%%%%%%%%%%%%%%%%%%%%%%%%%%%%%%%%%%%%%%%%%%%%%%%%%%%%%%%%
%%%%%%%%%%%%%%%%%%%%%%%%%%%%%%%%%%%%%%%%%%%%%%%%%%%%%%%%%%%%%%%%%%%%%%%%%%%%%%

\newpage

{\Huge \bfseries Micro/Meso-scopic view}
\addcontentsline{toc}{chapter}{Micro/Meso-scopic model}
\vspace{0.3cm}

One of the most useful ways of describing superfluid $\He$ at $T=0\unit{K}$ starts with nonlinear Schrodinger equation (NLSE) for the one-particle wave function $\psi$. Since the superfluid $\He$ is a strongly correlated system dominated by collective effects, this imperfect Bose-Einstein condensate (BEC) is approximately described by Gross-Pitaevskii equation (\ref{gross-pit}). Although, it must be noted that the real Helium-II is a dense fluid, not a weakly interacting Bose gas described by NLSE.

%%%%%%%%%%%%%%%%%%%%%%%%%%%%%%%%%%%%%%%%%%%%%%%%%%%%%%%%%%%%%%%%%%%%%%%%%%%%%%

\section{Gross-Pitaevskii model}

In terms of single-particle wavefunction $\psi(\vec{r},t)$ we write the Gross-Pitaevskii model:

\begin{equation}
i \hbar \frac{\partial \psi}{\partial t} = - \frac{\hbar^2}{2m} \nabla^2 \psi
+ \psi \int \vert \psi(\vec{r}^{\prime},t) \vert^2 V(\vert \vec{r} - \vec{r}^{\prime} \vert)
\text{d}\vec{r}^{\prime}\,,
\label{gross-pit}
\end{equation}

where $V(\vert \vec{r} - \vec{r}^{\prime} \vert)$ is the potential of two-body interaction between bosons. The normalization is set as $\int \vert \psi \vert^2 \text{d}\vec{r} = N$, where $N$ is number of bosons. By replacing potential with repulsive $\delta$-function of strength $V_0$ one obtains:

\begin{equation}
i \hbar \frac{\partial \Psi}{\partial t} = - \frac{\hbar^2}{2m} \nabla^2 \Psi - m\eps \Psi + V_0 \vert \Psi \vert^2 \Psi\,,
\label{GP}
\end{equation}

where $\eps$ is the energy per unit mass and $\Psi = A e^{i\Phi}$ is a macroscopic wave function of condensate. In this way one can define the condensate's density $\rho_{BEC} = m\Psi\Psi^* =  mA^2$ and velocity $\vec{v}_{BEC} = (\hbar / m)\nabla \Psi$. Note that equation (\ref{GP}) is equivalent to a continuity equation and an modified Euler equation (by the so called quantum pressure term).

Hereafter we identify $\rho_{BEC}$ with $\He$ superfluid component's $\rho_s$ at absolute zero and $\vec{v}_{BEC}$ with $\vec{v}_s$. It must be noted that this identification is convenient from the point of view of having a simple hydrodynamics model but is not entirely correct. The reason is
that Helium-II is a dense fluid, not the weakly interacting Bose gas described
by the NLSE (\ref{gross-pit}), so the condensate is not the same as the superfluid component.

\newpage

Even though the superfluid is irrotational: $\omega = \nabla \times \vec{v}_s = \vec{0}$, the NLSE has a vortex-like solution: $\vec{v}_s = \varkappa / 2\pi r\, \vec{e_{\theta}}$, where $\theta$ is the azimuthal angle and $\varkappa=9.97 \times 10^{-4} \unit{cm^2 \dotprod s^{-1}}$ is the \textit{quantum of circulation}, obtained from:

\begin{equation}
\varkappa = \oint_{\mathcal{C}} \vec{v}_{BEC} \cdot \unit{d}\vec{\boldsymbol{\ell}} = \frac{h}{m}\,,
\label{varkappa}
\end{equation}

where $\mathcal{C}$ is a closed loop surrounding the vortex core - a topological defect (\textbf{Figure {\ref{singularity}}}) within macroscopic wavefunction $\Psi$

\begin{figure}[h]
	\centering
	\includegraphics[width=0.6\textwidth]{graphics/theory/singularity}
	\caption{An illustration of topological singularities within a superfluid $\He$. Left: A singly-connected irrotational region with circulation along $\mathcal{C}$ loop equal to zero. Right: A multiply-connected region with depicted cores of quantized vortices with a finite circulation along $\mathcal{C}$ loop.}
	\label{singularity}
\end{figure}

%%%%%%%%%%%%%%%%%%%%%%%%%%%%%%%%%%%%%%%%%%%%%%%%%%%%%%%%%%%%%%%%%%%%%%%%%%%%%%

\section{Quantized vortex}

As Feynman showed \cite{feynman}, superfluid vortex lines appear when Helium-II moves faster than a certain critical velocity. The simplest way to create quantum vortices is to rotate cylinder with superfluid Helium-II with high enough angular velocity $\Omega$. Created vortex lines form an ordered array of density $L=2\Omega / \varkappa$, all aligned along the axis of rotation (\textbf{Figure \ref{rotating-helium}}). \textit{Vortex line density} $L$ can be also interpreted as a total vortex length in an unit volume.

\begin{figure}[h]
	\centering
	\includegraphics[width=0.4\textwidth]{graphics/theory/rotating-helium}
	\caption{Array of quantized vortices in a rotating container}
	\label{rotating-helium}
\end{figure}

The key properties of Onsager-Feynman vortex \cite{onsager} are the quantized circulation $\varkappa$, superfluid rotational velocity field $\vec{v}_s = \varkappa / 2\pi r\, \vec{e_{\theta}}$ and the \textit{vortex core parameter} $a_0$. The core size $a_0$ can be estimated by substituting $\vec{v}_s$ back into (\ref{GP}) and solving differential equation for $\rho_s$. One finds that $\rho_s$ tends to the value $m^2 \eps / V_0$ for $r \rightarrow \infty$ and to zero density for $r \rightarrow 0$.
The characteristic distance over which $\Psi$ collapses (superfluid density $\rho_s$ drops from bulk value to zero) is $a_0 \approx 10^{-10} \unit{m} = 1 \unit{\AA}$. From this, there is a conclusion that the vortex is hollow at its core and therefore, a topological defect occurs.

Taking $a_0$ as core radius, the energy considerations showed that a single vortex containing $N$ circulation quanta owns more energy than $N$ singly-quantized vortices. Hence it is generally assumed that only ground-state vortices are commonly observed.

Clearly, vortex lines don't have to be aligned in general. In most cases , the superfluid flow is strongly chaotic, better known as \textit{quantum turbulence}. This topic is covered in more detail later in this work.

%%%%%%%%%%%%%%%%%%%%%%%%%%%%%%%%%%%%%%%%%%%%%%%%%%%%%%%%%%%%%%%%%%%%%%%%%%%%%%

\section{Vortex filament model}

The vortex line can be represented as a curve via position vector $\vec{s} = \vec{s}(\xi, t)$ in three-dimensional space. Here, $\xi$ is arclength along the vortex line. Next we label $\vec{s}^{\prime}$ as $\text{d}\vec{s} / \text{d} \xi$ and $\vec{s}^{\prime\prime}$ as $\text{d}\vec{s}^{\prime} / \text{d} \xi$.
Within our context, $\vec{s}^{\prime}$ is a tangent vector and $\vert \vec{s}^{\prime\prime} \vert$ is a local curvature $R^{-1}$ at a given point.
The triad of vectors $\vec{s}^{\prime}$, $\vec{s}^{\prime\prime}$, $\vec{s}^{\prime} \times \vec{s}^{\prime\prime}$ are perpendicular to each other (\textbf{Figure \ref{filament}}) and point along the tangent, normal and binormal respectively:

\begin{figure}[h]
	\centering
	\includegraphics[width=0.7\textwidth]{graphics/theory/filament}
	\caption{Schematic of the vortex filament and the triad vectors $\vec{s}^{\prime}$, $\vec{s}^{\prime\prime}$, $\vec{s}^{\prime} \times \vec{s}^{\prime\prime}$. Source: \cite{tsubota}}
	\label{filament}
\end{figure}

We suppose that the superfluid component is incompressible $\nabla \dotprod \vec{v}_s = 0$. Moreover, superfluid vorticity $\omega_s$ is localized only at positions of vortex filament $\omega_s(\vec{r},t) = \nabla \times \vec{v}_s$. Combining these two properties gives the Poisson equation $\Delta \phi = \omega_s$ for the potential $\phi$.
Using Fourier transform one obtains \cite{barenghi} the Biot-Savart law for the superfluid velocity:

\begin{equation}
\vec{v}_s(\vec{r}) = \frac{\varkappa}{4\pi} \int_{\mathcal{L}} \frac{(\vec{r^{\prime}} - \vec{r}) \times \text{d}\vec{r^{\prime}}}{\vert \vec{r^{\prime}} - \vec{r} \vert^3}\,,
\label{biot_general}
\end{equation}

where the integral path $\mathcal{L}$ represents curves along all vortex filaments.

This law determines the superfluid velocity field $\vec{v}_s(\vec{r})$ via the arrangement of the vortex filaments. Now we define the \textit{self-induced} velocity $\vec{v}_{\text{ind}}$, describing the motion which a vortex line induces onto itself ($\vec{r} \rightarrow \vec{s}$ in (\ref{biot_general})) due to its own curvatures:

\begin{equation}
\vec{v}_{\text{ind}}(\vec{s}) = \frac{\varkappa}{4\pi} \int_{\mathcal{L}} \frac{(\vec{r^{\prime}} - \vec{s}) \times \text{d}\vec{r^{\prime}}}{\vert \vec{r^{\prime}} - \vec{s} \vert^3}
\label{biot_ind}
\end{equation}

However, this integral (\ref{biot_ind}) diverges as $\vec{r}^{\prime} \rightarrow \vec{s}$ because the core structure
of the quantized vortex was initially neglected. We avoid this divergence by splitting the integral into two parts - direct neighbourhood of the point $\vec{s}$ (local part) and the rest $\mathcal{L}^{\prime}$ (nonlocal part). The Taylor expansion of the local part leads \cite{schwarz} to a finite result:

\begin{align}
\vec{v}_{\text{ind}}(\vec{s})
= \vec{v}_{\text{ind,local}} + \vec{v}_{\text{ind,nonlocal}}
\approx& \beta \vec{s}^{\prime} \times \vec{s}^{\prime \prime} + \frac{\varkappa}{4\pi} \int_{\mathcal{L}^{\prime}} \frac{(\vec{r^{\prime}} - \vec{s}) \times \text{d}\vec{r^{\prime}}}{\vert \vec{r^{\prime}} - \vec{s} \vert^3}\,,
\label{lia+biot}
\\
\text{where}\, \beta =& \frac{\varkappa}{4\pi} \ln(R / a_0)\,,
\label{beta}
\end{align}

where $\mathcal{L}^{\prime}$ is the original vortex line without a close area of the studied vortex point and $R$ is a \textit{local curvature} and often is calculated as $1 / \vert \vec{s}^{\prime\prime} \vert$ \cite{schwarz}.

Since the local part of induced velocity (\ref{lia+biot}) is a dominant term (and also computationally faster), the nonlocal part can be neglected. Such approximation process is called as \textit{Local Induction Approximation} (LIA). LIA represents the contribution of local curvature to the induced velocity, whereas nonlocal Biot-Savart part represents the global filament curvature.

Since in the system there could be present also external flow sources of superfluid component (e.g. heat resistors causing \textit{counterflows}), we define the total superfluid velocity $\vec{v}_{s,tot}$, in a laboratory frame, as:

\begin{equation}
\vec{v}_{s,tot} = \vec{v}_{s,ext} + \vec{v}_{\text{ind}}
\end{equation}

\newpage

%%%%%%%%%%%%%%%%%%%%%%%%%%%%%%%%%%%%%%%%%%%%%%%%%%%%%%%%%%%%%%%%%%%%%%%%%%%%%%


\section{Vortex dynamics}

To determine the equation of motion of $\vec{s}(t)$ we recognize the forces acting upon the line - the magnus force $\vec{f}_M$ and (at non-zero temparature $T>0\unit{K}$ )the drag force $\vec{f}_D$ (both are per unit length).

The magnus force $\vec{f}_M$ always arises when a rotating body moves in a flow. This causes a pressure difference, which in our case of moving vortex line with circulation quantum $\varkappa$, exerts a force:

\begin{equation}
\vec{f}_M = \rho_s \varkappa \,\vec{s}^{\prime} \times (\vec{\dot{s}} - \vec{v}_{s,tot})\,,
\label{magnus}
\end{equation}

where $\vec{\dot{s}} = \text{d}\vec{s} / \text{d} t$ is the velocity of a particular point on a vortex line.

The drag force $\vec{f}_D$ arises from the \textit{mutual friction}, the interaction between the normal component and vortex lines (quantized superfluid component). According to the findings of Vinen and Hall \cite{vinen}, the normal fluid flowing with velocity $\vec{v}_n$ past a vortex core exerts a frictional force $\vec{f}_D$ on the vortex line, given by:

\begin{align}
\vec{f}_D = -& \alpha(T)\rho_s\varkappa\,\vec{s}^{\prime} \times [\vec{s}^{\prime} \times (\vec{v}_{ns} - \vec{v}_{\text{ind}})]
\label{alpha1}
\\
-& \alpha^{\prime}(T)\rho_s\varkappa\,\vec{s}^{\prime} \times (\vec{v}_{ns} - \vec{v}_{\text{ind}})
\,,
\label{alpha2}
\end{align}

where $\vec{v}_{ns} = \vec{v}_{n} - \vec{v}_{s,ext}$ is the difference between the average velocity of normal component and the applied superfluid velocity.

The temperature dependent dimensionless parameters $\alpha(T)$ and $\alpha^{\prime}(T)$ are often written in terms of measured \textit{mutual friction parameters} $B$ and $B^{\prime}$, which are known from experiments by Samuels and Donnelly \cite{donnelly}:

\begin{equation}
\alpha(T) = \frac{\rho_n B(T)}{2\rho}
\hspace{1cm}
\alpha^{\prime}(T) = \frac{\rho_n B^{\prime}(T)}{2\rho}
\end{equation}

The precise calculation of the mutual friction parameters $B(T), B^{\prime}(T)$ over the entire temperature range is still an open problem. Although, we already know that in the area of high temperatures, the friction arises mainly from the scattering processes of rotons.

\newpage

Since the mass of vortex core is usually neglected, the two forces $\vec{f}_M$ and $\vec{f}_D$ add up to zero: $\vec{f}_M + \vec{f}_D = \vec{0}$. Hence, solving for $\text{d}\vec{s} / \text{d} t$, we obtain \cite{schwarz} the Schwarz's equation:

\begin{equation}
\dot{s} = \vec{v}_{\text{s, ext}} + \vec{v}_{\text{ind}}
+ \alpha\vec{s}^{\prime} \times (\vec{v}_{ns} - \vec{v}_{\text{ind}})
- \alpha^{\prime}\vec{s}^{\prime} \times [\vec{s}^{\prime} \times (\vec{v}_{ns} - \vec{v}_{\text{ind}})]\,,
\label{schwarz}
\end{equation}

On the basis of Schwarz's equation (\ref{schwarz}), algorithms to numerically simulate vortex time evolution of an arbitrary configuration can be developed. Also, the vortex parametrisation $\vec{s}(\xi, t)$ and dynamics description provide the baseline of what we call as Vortex Filament (VF) model. More on VF model is written later in \textbf{Simulation} chapter.

%%%%%%%%%%%%%%%%%%%%%%%%%%%%%%%%%%%%%%%%%%%%%%%%%%%%%%%%%%%%%%%%%%%%%%%%%%%%%%


\subsection*{Quantized vortex rings}

A special case of vortex line configuration are a freely moving vortex rings. Such rings are usually created as a result of multi-vortex interconnections \cite{vortex_ring} or by the self-reconnection of an oscillating loop pinned to the surface of a vibrating body. The exact expressions derived from the Gross-Pitaevskii equation (\ref{gross-pit}) \cite{roberts} for the energy $E$ and induced center velocity $v_{\text{ind}}$ of a vortex ring, moving in a Helium-II of density $\rho$ and having a radius $R$ much greater than its core radius $R >> a_0$, are:

\begin{equation}
E = \frac{1}{2}\varkappa^2 \rho R \Big(\ln(8R/a_0) - 2 + c\Big)
\label{ring-energy}
\end{equation}

\begin{equation}
v_{\text{ind}} = \frac{\varkappa}{4\pi R} \Big(\ln(8R/a_0) - 1 + c\Big)\,,
\label{ring-velocity}
\end{equation}

where $c$ is a constant based on inner structure of the vortex. Since we work with hollow core, we use \cite{donnelly_book}
$c = 1/2$. Note that (\ref{ring-energy}) and (\ref{ring-velocity}) depend on $a_0$ only logarithmically.
The behavior of the vortex ring is thus quite insensitive to the exact value of $a_0$ (expected to be of the order of atomic dimension).

Relations (\ref{ring-energy}) and (\ref{ring-velocity}) are derived directly from Gross-Pitaevskii description and no dissipative process (mutual friction) was included. Therefore, the relations hold only for temperature $T=0\unit{K}$, Using the explicit dynamical equation \cite{donnelly_book} for vortex ring motion, one can also derive the final ring center velocity $\vec{v}_{\text{ring}}$ and energy $E_{\text{ring}}$ using (\ref{ring-energy}) and (\ref{ring-velocity}) like:

\begin{equation}
\vec{v}_{\text{ring}} = (1 - \alpha^{\prime}) (\vec{v}_{\text{ind}} - \vec{v}_{\text{s, ext}})
+ \alpha^{\prime} \vec{v}_{\text{n, ext}}
\label{v_ring}
\end{equation}

\begin{equation}
E_{\text{ring}} = \Big( \frac{\alpha}{1 - \alpha^{\prime}} \Big) E
\label{E_ring}
\end{equation}

Several other interesting results come from the ring's dynamic motion equation and the mutual friction formula (\ref{alpha1}), (\ref{alpha2}). The second term (\ref{alpha2}) of friction force causes the decrease of vortex ring radius, whereas the first term (\ref{alpha1}) is the dissipative term. The superfluid vortex ring (\textbf{Figure (\ref{vortex-ring})}) living in a mixture of a normal and superfluid component has therefore a limited lifetime expectancy and the travelled distance.

\begin{figure}[h]
	\centering
	\includegraphics[width=0.4\textwidth]{graphics/theory/vortex-ring}
	\caption{Depiction of quantized vortex ring motion and induced velocity field. Source: Huang, Kerson, \textit{Quantum vorticity in nature}, arXiv.}
	\label{vortex-ring}
\end{figure}

More explicitly, it was shown \cite{donnelly_book} that in case of weak counterflow velocity $\vec{v}_{\text{ns}}$, the lifetime of vortex ring can be expressed as a simple relation:

\begin{equation}
\tau_{\text{ring}} = \frac{R_0}{2 \alpha \vert \vec{v}_{\text{ring}}(R_0)\vert}\,,
\label{ring-lifetime}
\end{equation}

where $R_0$ is the initial radius of created vortex ring.

By integrating the ring's motion equation from $R_0$ to $R(\tau) \approx a_0$ we obtain the distance travelled by the ring's center:

\begin{equation}
D_{\text{ring}} = \frac{\alpha}{1 - \alpha^{\prime}} (R_0 - a_0)
\label{ring-distance}
\end{equation}

Relations (\ref{ring-lifetime}) and (\ref{ring-distance}) are taken as a baseline in \textbf{Simulation} chapter.
\newpage

%%%%%%%%%%%%%%%%%%%%%%%%%%%%%%%%%%%%%%%%%%%%%%%%%%%%%%%%%%%%%%%%%%%%%%%%%%%%%%%%%%%%%%%%%%%%%%%%%%%%%%%%%%%%%%%%%%%%%%%%%%%%%%%%%%%%%%%%%%%%%%%%%%%%%%%%%%%%%%%%%%%%%%%%%%%%%%%%%%%%%%%%%%%%%%%%%%%%%%%%%%%%%%%%%%%%%%%%%%%%%%%%%%%%%%%%%%

{\Huge \bfseries Macroscopic view}
\addcontentsline{toc}{chapter}{Macroscopic model}
\vspace{0.3cm}

Besides NLSE and Vortex filament model, there is also a third, \textit{macroscopic} model in which the individual vortex lines are \textit{invisible} and the superfuid component of Helium-II is considered as a continuous flow of vortices. Many phenomena are similar to those in classical hydrodynamics (\textbf{Figure \ref{laminar-turbulent}}), but there emerge also new and very special type of events that can happen within superfluid Helium-II.

\begin{figure}[h]
	\centering
	\includegraphics[width=0.5\textwidth]{graphics/theory/laminar-turbulent}
	\caption{Depiction of a classical steady flow for various Reynolds number values. Many phenomena of laminar, semi-turbulent and turbulent flows are visible in depcitions. Source: \cite{laminar-turbulence}}
	\label{laminar-turbulent}
\end{figure}

%%%%%%%%%%%%%%%%%%%%%%%%%%%%%%%%%%%%%%%%%%%%%%%%%%%%%%%%%%%%%%%%%%%%%%%%%%%%%%%%%%%%%%%%%%%%%%%%%%%%%%%%%%%%%%%%%%%%%%%%%%%%%%%%%%%%%%%%%%%%%%%%%%%%%%%%%%%%

\section{Hydrodynamics of superfluid}

Such macroscopic model is called the Hall-Vinen-Bekarevich-Khalatnikov (HVBK) model and provides a generalization of Landau's two-fluid model equations, including the presence of vortices. The superfluid is treated as a continuum and we can define a macroscopic superfluid vorticity $\vec{\Omega}_s$, despite the fact that, microscopically, the superfluid velocity field obeys $\nabla \times \vec{v}_s = \vec{0}$. The downside of this model is its assumption of spatially (not randomly) organized vortices. The common example is a rotating cylinder \cite{osborne}.\\

\newpage

The incompressible HVBK equations for normal component $\vec{v}_n (\vec{r}, t)$ and superfluid component $\vec{v}_s (\vec{r}, t)$, respectively, are \cite{barenghi}:

\begin{align}
\frac{\partial\vec{v}_n}{\partial t} + (\vec{v}_n\cdot \nabla)\vec{v}_n =& -\frac{1}{\rho} \nabla P - \frac{\rho_s}{\rho_n} S \nabla T + \frac{\eta}{\rho_n} \nabla^2 \vec{v}_n + \vec{F}_{ns}\,,
\label{motion_normal}\\
\frac{\partial\vec{v}_s}{\partial t} + (\vec{v}_s\cdot \nabla)\vec{v}_s =& -\frac{1}{\rho} \nabla P + S \nabla T + \vec{T} - \frac{\rho_n}{\rho} \vec{F}_{ns}\,,
\hspace{15mm}
\label{motion_super}
\end{align}

where we have defined:

\begin{equation}
\vec{F_{ns}} = \frac{B(T)}{2} \vec{\hat{\Omega}} \times [\vec{\hat{\Omega}}_s \times (\vec{v}_n - \vec{v}_s - \nu_s\nabla \times \vec{\hat{\Omega}})]
+ \frac{B^{\prime}(T)}{2} \vec{\Omega}_s \times (\vec{v}_n - \vec{v}_s - \nu_s\nabla \times \vec{\hat{\Omega}}_s)\,,
\end{equation}
\begin{equation}
\vec{\Omega}_s = \nabla \times \vec{v}_s\,,
\end{equation}
\begin{equation}
\vec{\hat{\Omega}}_s = \vec{\Omega}_s / \vert \vec{\Omega}_s \vert\,,
\end{equation}
\begin{equation}
\vec{T} = - \frac{\varkappa}{4\pi} \log(b_0 / a_0) \, \vec{\Omega}_s \times (\nabla \times \vec{\hat{\Omega}}_s)
\end{equation}

Here we can identify the quantities as $\vec{F_{ns}}$ (mutual friction force), $\vec{T}$ (vortex tension) and $\eta$ (viscosity parameter). Usually, $b_0$ is the intervortex spacing and can be estimated as $b_0 = (2\Omega_s \varkappa)^{-1/2}$. %Note that two-fluid equations (\ref{motion_normal}), (\ref{motion_super}) by Landau can be achieved by neglecting $\vec{F}$ and $\vec{T}$.
The HVBK equations have well-defined limiting cases:

\begin{itemize}
	\item $T \rightarrow T_{\lambda}$: In this case $\rho_s \rightarrow 0$ and the normal fluid motion equation (\ref{motion_normal}) becomes the classical Navier-Stokes equation with viscosity term.

	\item $T \rightarrow 0$: In this case $\rho_n \rightarrow 0$ and the superfluid motion equation (\ref{motion_super}) describes a pure (potential) superflow. Additionally, taking the classical limit ($\hbar \rightarrow 0$) would give us the pure Euler equation of inviscid fluid.
\end{itemize}

The HVBK model has been widely used with success to study the transition to classical or quantum turbulence, for estimations of critical Reynolds numbers and its temperature dependence.

\newpage

%%%%%%%%%%%%%%%%%%%%%%%%%%%%%%%%%%%%%%%%%%%%%%%%%%%%%%%%%%%%%%%%%%%%%%%%%%%%%%%%%%%%%%%%%%%%%%%%%%%%%%%%%%%%%%%%%%%%%%%%%%%%%%%%%%%%%%%%%%%%%%%%%%%%%%%%%%%%

\section{Dynamical similarity principle}

An important role in the behaviour of fluids is taken by the \textit{fluid dimensional numbers}, which are used for scalling of motion equations in fluid mechanics.

The principle of \textit{dynamical similarity} states that two flows of similar geometry have the same dynamical behaviour, if they can be characterised by the same set of suitable dimensionless parameters representing transport phenomena. In order to describe Helium-II with correct equations and with the most precision, we have to choose which dimensionless parameters are useful.

\begin{itemize}
	\item \underline{Knudsen number} (Kn): This number helps determine whether statistical mechanics or the continuum mechanics formulation of fluid should be used to model the system. Kn is defined as the ratio of the molecular mean free path $\lambda$ to a representative physical length scale $D$ (container size).\\
	If the temperature of Helium-II is set above $T > 1.0 \unit{K}$, there is a still sufficient amount of normal component and the mean free path of thermal excitations is much smaller, comparing it with container scale $\lambda \ll D$. In that case, continuum mechanics could be used as a macroscopic theory for superfluid Helium-II.

	However, if temperature is below $T < 0.6 \unit{K}$, the mean free path $\lambda$ becomes comparable with length scale $D$ and the continuum model starts to break down. Here, the system is rather described as a gas of thermal excitations.

	\item \underline{Weissenberg number} (Wi): This number relates the typical frequency of perturbations, $\omega$ , of the fluid with the characteristic time, $\tau$ that describes the relaxation of the fluid towards a thermodynamic equilibrium. The Weissenberg number is then given as a multiplication of oscillation frequency $\omega$ and the relaxation time $\tau$.\\
	Since the relaxation time of Helium-II is relatively small in temperatures above $T > 1\unit{K}$, then $\text{Wi} = \omega \tau \ll 1$, so Helium-II can be considered as a Newtonian fluid.

	Once again, if temperature is below $T < 0.6 \unit{K}$, relaxation time of thermal excitations rises rapidly, meaning $\text{Wi} \sim 1$, causing the excitations propagating ballistically.

	\item \underline{Reynolds number} (Re): Let's consider the continuum and newtonian assumptions ($\text{Kn} \ll 1$ and $\text{Wi} \ll 1$), so the fluid can be described by the raw form of Navier-Stokes (N-S) equation of motion. When we take into account also the incompressibility $\nabla \dotprod \vec{v} = \vec{0}$, the N-S for classical fluid reduces itself to its most simplest form.\\
	In case of stationary flow ($\partial \vec{v} / \partial t = \vec{0}$), N-S can be rewritten into a dimensionless form. Following these steps, there arises typical values of velocity $U$ and length scale $L$, at which there is the most significant change in velocity. Re can be expressed as a ratio of inertial and dissipative forces as $\text{Re} = U L \rho / \eta$, where $\eta$ is the dynamic viscosity of the flow field.\\
	The oscillatory case is described more in the next section.

\end{itemize}

The derivation of the \textit{dynamical similarity} phenomenon can be directly seen from inspection of the underlying motion equation (\ref{motion_normal}) with geometrically similar bodies. In the classical fluid dynamics, we use dynamical similarity and scaling arguments for expressing experimental data in terms of Reynolds numbers, drag coefficients, lift, and so on.

Alternatively, the dissipative forces may be described in terms of a dimensionless \textit{drag coefficient} $C_D$, representing the relation between \textit{drag force} $\vec{F}$ and fluid velocity $\vec{U}$, and usually takes the form:

$$
C_D \!\propto\! U^\alpha, \hspace{1cm}
\text{where}
\left\{
  \begin{array}{l l}
    \alpha=-1 & \quad \text{for laminar flow $\Re \in (0-10)$}\\
    \alpha=0 & \quad \text{for turbulent flow $\Re \in (10^3-10^5)$}
  \end{array}
\right.\,,
$$

where the first case ($C_D \propto 1/U$) represents the viscous skin friction anf the second case ($C_D \approx \text{const.}$) represents the pressu drag.\\
It is very common to plot experimentally measured dependence of drag coefficient $C_D$ against the Reynolds number, for various objects (sphere, disc, cylinder) past flow:

\begin{figure}[h]
	\centering
	\includegraphics[width=0.7\textwidth]{graphics/theory/C-Re}
	\caption{Drag coefficients of different objects in a steady flow with changing Reynlods number. Blue line - thin disc, Green line - cylinder with drag crisis near $\Re \approx 10^6$, Red line - sphere with similar drag crisis as with cylinder, Black dashed line - laminar drag, where $C_D \propto \Re^{-1}$.}
	\label{C-Re}
\end{figure}

Dynamical similarity argument also leads to the existence of critical Reynolds number, at which the transition to turbulence in case of classical fluid occurs. Note that since superfluid Helium-II is composed of two fluids, the mentioned applies only to the normal component. The transition of superfluid component to quantum turbulence is described wider in next chapters

%%%%%%%%%%%%%%%%%%%%%%%%%%%%%%%%%%%%%%%%%%%%%%%%%%%%%%%%%%%%%%%%%%%%%%%%%%%%%%%%%%%%%%%%%%%%%%%%%%%%%%%%%%%%%%%%%%%%%%%%%%%%%%%%%%%%%%%%%%%%%%%%%%%%%%%%%%%%

\section{Oscillatory flows}

If a given body is oscillating in a classical viscous fluid, described by ordinary Navier-Stokes equation, there appears another characteristic lenght scale, identified by \cite{landau} as the \textit{viscous penetration depth}:

\begin{equation}
\delta(\omega) = \sqrt{\frac{2\eta}{\rho\omega}}\,,
\label{penetration}
\end{equation}

where $\omega$ is the angular frequency of oscilations.\\
To recognize correct characteristic length scale (whether it should be oscillating body dimension $D$ or penetration depth $\delta$), we calculate the ratio of time-dependent term $\partial \vec{v} / \partial t$ from N-S equation (\ref{motion_normal}) to the viscous term $\nabla^2 \vec{v}$. Hence, we define another dimensionless quantity, the \underline{Stokes number} $\beta$.
We calculate it \cite{stokes} as $\beta = \omega \rho D^2 / (\pi \eta)$, which can be reduced using (\ref{penetration}) to $\beta = D^2 / (\pi \delta^2)$.\\
From this, we call a situation as the \textit{high-frequency regime}, when $\delta \ll D$, so the Stokes values are $\beta \gg 1$.

\subsection*{Classical hydrodynamics}

To describe fully an oscillatory flow, the governing Navier-Stokes equations may be expressed in terms of dimensionless velocity $U$, time $T$ and positions $L$. The independent time scale is given by the angular frequency of oscillations $\omega$. Candidates for characteristic lenght scale $L$ may lead to the body size $D$, surface roughness, or the viscous penetration depth $\delta(\omega)$.

In the high frequency limit (directly from \ref{penetration}) $\omega \gg 2\eta / (\rho D^2)$, depending on body geometry, one can reach $\delta(\omega) \ll D$ and say that fluid oscillates in
high Stokes regime \cite{universal_scaling} with $\beta = D^2 / (\pi \delta^2) \gg 1 $. Also, when also the surface roughness exceeds $\delta$, the N-S equation may be expressed using only one dimensionless parameter: an \textit{oscillatory Reynolds number} $\Re_{\delta} = \delta U \rho / \eta$.

\subsection*{Superfluid Helium-II}

Assuming two independent velocity fields $\vec{v}_n (\vec{r}, t)$, $\vec{v}_s (\vec{r}, t)$ in superfluid Helium-II, the above thoughts are applicable for the oscillatory viscous flow of the normal component $\vec{v}_n$. We therefore define the high frequency limit for normal component as:

\begin{equation}
\delta_n = \sqrt{\frac{2\eta}{\rho_n\omega}}\,,
\hspace{1cm}
\text{Dn} = \frac{U \delta_n \rho_n}{\eta}
\label{donnelly}\,,
\end{equation}

We will call the oscillatory Reynolds number for normal component in superfluid Helium-II as a \textit{Donnelly number} (Dn, after Russell J. Donnelly, who as first came with this (\ref{donnelly}) idea.

At low velocities, below the critical thresholds to create either classical or quantum turbulence, the flow of the superfluid component is purely potential and the normal component exhibits laminar viscous flow.

If the typical body curvature is of order $1/D$, the surface may be described as if consisting of many planar elements. In this case it is shown \cite{universal_scaling} that we can write for the average dissipated energy:

\begin{equation}
\langle \dot{E} \rangle =
\frac{1}{2} \alpha \xi U_0^2 S_{eff} \sqrt{\frac{\eta \rho_n \omega}{2}}\,,
\label{energy_loss}
\end{equation}

where $U_0$ is the velocity amplitude, $S_{eff}$ the effective surface, $\alpha$ the mutual friction constant and $\xi$ the integral of velocity profile along the resonator. The total energy of an oscillator is given as $E = \xi m U_0^2 / 2$ and moreover, we define a fluidic quality factor $Q_f$ of an oscillator for a single oscillation as:

\begin{equation}
1 / Q_f = \frac{\langle \dot{E} \rangle}{\omega E} = \frac{\alpha \rho_n S_{eff} \delta_n}{2m}
\label{quality}
\end{equation}

From (\ref{energy_loss}), one can also derive the peak dissipative force (during a period of oscillation) and the dimensionless drag coefficient related to the normal component:

\begin{equation}
F_0 = \frac{2 \omega \langle \dot{E} \rangle}{U_0}
= \alpha \xi \omega S_{eff} U_0 \sqrt{\frac{\eta \rho_n \omega}{2}}
\hspace{5mm}
\rightarrow
\hspace{5mm}
C_D^{\,n} = \frac{2 F}{A \rho_n U_0^2} = \frac{2 S_{eff}}{A U_0} \sqrt{\frac{\eta \omega}{\rho_n}}\,,
\label{drag_normal}
\end{equation}

where $A$ is the cross-section of the body perpendicular to flow. According to dynamical similarity principle, the drag coefficient (\ref{drag_normal}) can be expressed as a function of the dimensionless Donnelly number (\ref{donnelly}):

\begin{equation}
C_D^{\,n} = \Phi / \text{Dn}\,,
\end{equation}

where number $\Phi$ is determined by the oscillating body geometry. Clearly, the laminar case of normal component is fully described by the hydrodynamic laws. In turbulent case, we expect a constant value for $C_D^{\,n}$ as long, as only normal component contributes to the drag force (no quantum turbulence).

\section{Quantum turbulence}

Turbulence of superfluid component can be viewed as a tangle of vortex lines. In this case quantized vortices can be nucleated either \textit{intrinsically} (such process requires critical velocities of order $10 \unit{m/s}$) or \textit{extrinsically}, by stretching and reconnections of seed vortex loops. The initial vortices in the extrinsical case are the remnant vortices, which always exist in macroscopic samples of Helium-II . In many types of flow the critical velocity for extrinsic vortex nucleation is observed to be in order $\sim\unit{cm/s}$.

The superfluid component becomes turbulent at some critical velocity $U_C$ and therefore, we expect an increase in drag coefficient $C_D$ much above the possible dependence caused by turbulence of normal component. The process of self-reconecting remnant vortices was well studied \cite{universal_scaling} and the related critical velocity is expected to scale as $U_C \propto \sqrt{\varkappa \omega}$, where $\varkappa$ is the circulation quantum. Hence, it is convenient to define a dimensionless velocity $\hat{U}$ with related drag coefficient $C_D^{\,s}$ as:

\begin{equation}
\hat{U} = U_0 / \sqrt{\varkappa \omega}
\hspace{5mm}
\rightarrow
\hspace{5mm}
C_D^{\,s} = \frac{2 F_0}{A \rho_s U_0^2} = \frac{2 F_0}{A \rho_s \varkappa \omega \hat{U}^2}\,,
\label{drag_super}
\end{equation}

In case of turbulent superfluid component with velocities sufficiently above critical values, we expect both component to be coupled due to the mutual friction. In this case, both components contribute to the pressure drag, and drag coefficient must be calculated classicaly as $C_D = 2F / (A\rho U^2)$. where $\rho$ is the total density of Helium-II.

\newpage

\subsection*{Ultra-low temperature regime}

In classical fluids, when the mean free path $\lambda$ of particles becomes comparable to a lenght scale $D$ of the system ($\text{Kn} \sim 1$), the continuum model of the fluid starts to break down and the system cannot longer be described by the Navier-Stokes equations. Similar arguments go when the angular frequency of oscillatory flow $\omega$ becomes comparable with the relaxation time $\tau$ of the fluid towards thermodynamic equilibrium ($\text{Wi} = \omega \tau \sim 1$).\\
Here, the system is described as a gas of thermal quasiparticles propagating
ballistically through a physical vacuum. Therefore, such system state is called as \textit{ballistic regime}.

In practice with superfluid Helium-II, such situation is usually reached by cooling fluid  down to ultra-low temperatures. Below $T < 1\unit{K}$, the normal component accounts for less than $1\%$ of the total density, but required cooling to obtain ballistic regime (\textbf{Figure (\ref{ballistic})}) is below $T < 0.6\unit{K}$ (here the Helium-II is better described as a gas of thermal quasiparticles than a fluid).

\begin{figure}[h]
	\centering
	\includegraphics[width=0.7\textwidth]{graphics/theory/ballistic}
	\caption{Phonon Knudsen number and Weissenberg numbers plotted against temperature for different oscillators. The dashed line separates the interval $T < 0.6\unit{K}$, where the ballistic regime takes a place. Source: \cite{universal_scaling}}
	\label{ballistic}
\end{figure}

\newpage

\subsection*{Universal scaling}

A study was conducted in order to solve the Stokes' second problem with an oscillating plane in viscous fluid, using pure Boltzmann kinetic equation. This solution is used to derive a universality relation valid in the high frequency limit (with no turbulence present) across both Newtonian and non-Newtonian regimes of the fluid. Using scaling function $f(\omega \tau)$, the authors derived \cite{universal_scaling} the relation for the quality factor:

\begin{equation}
1 / Q_f = \frac{\alpha S_{eff}}{m} \sqrt{\frac{\eta \rho_n}{2\omega}} f(\text{Wi})
\end{equation}

In \textbf{Results} we use this form of universality scaling for comparison against experimental data collected in temperature ranges $T < 0.6\unit{K}$ or $T > 1\unit{K}$.

\subsection*{Multiple critical velocites}

Here is briefly commented the transition to quantum turbulence regime observed at very low temperatures ($T \ll 1\unit{K}$). A couple of experimental studies in milliKelvin temperatures reported \cite{crit-velocity} the existence of more critical velocities related with superfluid component flow within single experiment:

\begin{itemize}
	\item \underline{First critical velocity} is related to the formation of pinned vortex loops at the surface of oscillating body - possibly forming a thin layer with different hydrodynamic behaviour which increases the effective mass of the oscillating object. Such critical velocity is hard to observe at higher than ultra-low temperatures.

	\item \underline{Second critical velocity} is a consequence of vortex rings escaping from the oscillator body into the superfluid bulk and eventually forming a vortex tangle. Here the sudden raise of drag is observed, usually with hysteresis effect.

	\item \underline{Third critical velocity} is the highest critical velocity which can be observed. The origin of this velocity is linked to the development of larger quantized vortex structures, which in larger scale start to mimic the classical turbulence. Such velocity is in order $\approx \unit{m/s}$ and moreover, very likely screened by the influence of the normal component turbulence. Therefore, not likely reachable within experiments reported in this Rhesis.
\end{itemize}

% When both velocities of Helium-II are high enough, we expect the turbulent regime on both sides to be coupled due to mutual friction and contributing to the drag. In such situation, we are forced to use classical hydrodynamic metrics: drag coefficient $C_D = 2F / (A\rho U^2)$ and Donnelly number $D = U \delta / \nu$. Recent researches also hint that both classical turbulent and quantum turbulent regimes can exist separately with low interaction.
%
% Note that all presented approches are only approximate, since they are neglecting flows near the oscillating body, evaporation processes, sound emissions and other corner-case effects.

% \section{Second sound}
%
% Ordinary sound (the wave of density $\rho$ and pressure $P$) in Helium-II is called \textit{first sound}. In such process, temperature $T$ and entropy $S$ is conserved and $\vec{v}_n$ and $\vec{v}_s$ oscillate in phase with each other ($\vec{v}_n(t) \approx \vec{v}_s(t)$). On the contrary, by combination of two-fluid motion and continuity equations, one obtains the wave equation also for the temperature $T$ and entropy $S$. In such case the velocities obey an antiphase oscillation ($\rho_n(t) \vec{v}_n(t) \approx - \rho_s(t) \vec{v}_s(t)$) and remains $\rho$ and $P$ constant.
%
% \begin{figure}[h]
% 	\centering
% 	\includegraphics[width=0.99\textwidth]{graphics/theory/ss_1}
% 	\caption{here will come more proper picture}
% 	\label{ss_1}
% \end{figure}
%
% \begin{figure}[h]
% 	\centering
% 	\includegraphics[width=0.99\textwidth]{graphics/theory/ss_2}
% 	\caption{here will come more proper picture}
% 	\label{ss_2}
% \end{figure}
%
% In early Vinen's observations, the exponentially damped velocity of second sound wave, propagating through two-fluid medium, was derived from motion equations:
%
% \begin{equation}
% \vec{v}_{\ind{ns}} \propto e^{-\alpha \vert \vec{r} \vert} \vec{\hat{e}}_{\vec{r}}(\vec{k},\vec{r},\omega,t)\,,
% \end{equation}
%
% where $\alpha$ is the attenuation coefficient. When considering homogeneous chaotic distribution of vortex tangle, one can also derive the formula for the attenuation factor:
%
% \begin{equation}
% \alpha = \frac{B\kappa L}{6 \vert \vec{c}_2\vert}
% \label{alpha_mean}\,,
% \end{equation}
%
% where $B$ is the first mutual friction parameter and $\vec{c_2}$ is the initial second sound velocity. This velocity was experimentally examined and there was found a plateau near the value $20 \unit{m/s}$, which is desirable during experiments (stability against small temperature changes):
%
% \begin{figure}[h]
% 	\centering
% 	\includegraphics[width=0.4\textwidth]{graphics/theory/ss_velocity}
% 	\caption{velocity of the second sound with temperature}
% 	\label{ss_velocity}
% \end{figure}
%
% In this work, the phenomenon of second sound attenuation is used for detection of quantized vortices, which naturally appear within Helium-II. There is written much more about the method itself within the Experimental Approach part.

\newpage
