\documentclass[a4paper, 12pt]{report}

% Margins: left 40mm, right 25mm, up and down 25mm
% Dimensions: 210mm x 297mm
% (warning, Latex default from left: 1in~25mm)

\usepackage{fullpage}
\addtolength{\hoffset}{0cm} % extends height of page
\addtolength{\textwidth}{0.5cm} % extends lenght of written text

% Everything about page geometry:
%https://www.sharelatex.com/learn/Page_size_and_margins

%% PRINTED VERSION
% uncomment this below to render text only on pages on the right side - for printed version
%\let\openright=\clearpage

%% ENCODINGS
\usepackage[utf8]{inputenc}
%\usepackage[IL2]{fontenc}
\usepackage[english]{babel}


%% PACKAGES
\usepackage{amsfonts}
\usepackage{amsmath}
\usepackage{amssymb}
\usepackage[font={footnotesize}]{caption}
\usepackage{csquotes}
\usepackage{enumerate}
\usepackage{epigraph}
\usepackage{graphicx}
\usepackage{setspace} % setup various spaces
\usepackage{titlesec}
\usepackage[titles]{tocloft}
\usepackage[labelfont={sf,bf}]{caption}
\usepackage{url}
\usepackage{wrapfig} % image wrapper
\usepackage{hyperref} % hyperlinks


%% MACROS
\newcommand{\unit}[1]{\,\mathrm{#1}} % nice units
\newcommand{\ind}[1]{\mathrm{#1}}
\newcommand{\He}{{}^4\mathrm{He}} % special Helium
\renewcommand{\Re}{\mathrm{Re}} % Reynolds number
\newcommand{\eps}{\varepsilon} % standard epsilon
\newcommand{\<}{\langle} % closed left
\renewcommand{\>}{\rangle} % closed right
\renewcommand{\vec}[1]{\mathbf{#1}} % nice bold vector
\renewcommand{\dot}{\!\cdot\!} % scalar product
\newcommand{\todo}[1]{ {\bf !!TODO!!}\qquad{#1} } % TODO marker

% optimal spacings
\setlength{\parskip}{9pt}
\setlength{\parindent}{0pt}
\setlength{\baselineskip}{1.2\baselineskip} % extend distance under equations

% The following part convinces Latex to not mess with the chapters...
\makeatletter
\def\@makechapterhead#1{
  {\parindent \z@ \raggedright \normalfont
   \Huge\bfseries \thechapter. #1
   \par\nobreak
   \vskip 20\p@
}}
\def\@makeschapterhead#1{
  {\parindent \z@ \raggedright \normalfont
   \Huge\bfseries #1
   \par\nobreak
   \vskip 20\p@
}}\newpage
\makeatother

% Defining a non-numbered chapter (but included in Summary)
\def\chapwithtoc#1{
\chapter*{#1}
\addcontentsline{toc}{chapter}{#1}
}

% Changing font of subsection
\titleformat{\subsection}
{\large\sffamily\bfseries}
{\thesubsection}{1.4em}{}

\begin{document}

\pagenumbering{arabic}
\pagestyle{plain}
\setcounter{page}{1}

% Content
\tableofcontents


\chapter*{Introduction / Motivation (3 pgs)}
\addcontentsline{toc}{chapter}{Introduction}

\begin{itemize}
	\item helium backround, phase diagram, superfluid transition, interesting properties
	\item quantum turbulence, comparison with classical turbulence
	\item overview on experiments with QT
	\item overview on numerical simulations of QT
	\item motivations, goals of this work
\end{itemize}

\newpage
\chapter{Theoretical Background (15 pgs)}

The theoretical part of this Thesis is composed of three chapters:

\begin{itemize}
	\item[1.] Microscopic view - wave function, collapse, quantum vortices

	\item[2.] Mesoscopic view - filaments, drag and magnus force, dynamics of vortex, Schwarz equation, Kelvin waves (?)

	\item[3.] Macroscopic view - hydrodynamics, two-fluid model, 

\end{itemize}
\vspace{1cm}
{\Huge \bfseries Microscopic model}
\addcontentsline{toc}{chapter}{Microscopic model}
\vspace{0.3cm}

\section{Bose-Einstein condensate}
London's theory, macroscopic wave function, vorticity

\section{Quantum vortex properties}
quantized circulation, properties of quantum vortex (velocity field, energy)


{\Huge \bfseries Mesoscopic model}
\addcontentsline{toc}{chapter}{Mesoscopic model}
\vspace{0.3cm}

\section{Filaments theory}

segments, biot-savart, local induction approximation

\section{Filaments dynamics}

{\Huge \bfseries Macroscopic model}
\addcontentsline{toc}{chapter}{Macroscopic model}
\vspace{0.3cm}

hydrodynamics

\section{Two-fluid model}

second sound, mutual friction


\section{Oscillatory motion}

oscillatory flow, 

\section{Quantum turbulence}



\end{document}